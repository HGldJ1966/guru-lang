\documentclass[9pt,natbib]{sigplanconf}

%\usepackage{latex8}
%\usepackage{times}
\usepackage{amsmath}
\usepackage{latexsym}
\usepackage{amssymb}
\usepackage{proof}
\usepackage{url}

% \usepackage{natbib}
%  \bibpunct();A{},
%  \let\cite=\citep

\pdfpagewidth=8.5in
\pdfpageheight=11in

\newcommand{\lam}[2]{\lambda #1 . #2}
\newcommand{\lams}[2]{\lambda^* #1 . #2}
\newcommand{\alet}[3]{\textit{let}\ #1\ :=\ #2\ \textit{in}\ #3}
\newcommand{\Thet}[1]{\lam{V}{\lam{L}{\lam{A}{\lam{O}{\lam{C}{\lam{S}{\lam{D}{#1}}}}}}}}
\newcommand{\ope}[0]{\textit{open}}
\newcommand{\swap}[0]{\textit{swap}}
\newcommand{\vcomp}[0]{\textit{vcomp}}
\newcommand{\nlambda}[0]{\bar{\lambda}}
\newcommand{\nlam}[2]{\nlambda #1 . #2}
\newcommand{\rulename}[1]{\text{\textup{\textsf{#1}}}}
\newcommand\bs{\char '134 }  % A backslash character for \tt font
\newcommand{\seq}[3]{#1 \vdash #2 : #3}
\newcommand{\eval}[0]{\Downarrow}
\newcommand{\evalj}[2]{#1\, \eval\, #2}
\newcommand{\starstar}[0]{*\negthinspace*}
\newcommand{\nat}[0]{\mathbb{N}}
\newcommand{\optt}{\textsc{OpTT}}

\newcommand{\rase}[1]{\ulcorner #1 \urcorner}
\newcommand{\lowr}[1]{\llcorner #1 \lrcorner}

\newcommand{\Eq}[0]{\texttt{=}}
\newcommand{\Neq}[0]{\texttt{!=}}
\newcommand{\Qeq}[0]{\stackrel{?}{=}}
\newcommand{\bang}[0]{\texttt{!}}
\newcommand{\quant}[0]{\textit{Quant}}

\newtheorem{theorem}{Theorem}
\newtheorem{proposition}{Proposition}
\newtheorem{definition}{Definition}
\newtheorem{lemma}{Lemma}

\newcommand{\To}{\Rightarrow}
\newcommand{\gtrans}[2]{#1 \cdot #2}
\newcommand{\gtransa}[2]{#1 \cdot_1 #2}
\newcommand{\gtransb}[2]{#1 \cdot_2 #2}
\newcommand{\gsymm}[1]{#1^{-1}}
\newcommand{\gcong}[1]{f(#1)}
\newcommand{\ginj}[1]{f^{-1}(#1)}
\newcommand{\guru}[0]{\textsc{Guru}}

\begin{document}

\conferenceinfo{PLPV'10,} {January 19, 2010, Madrid, Spain.}
\CopyrightYear{2010}
\copyrightdata{978-1-60558-890-2/10/01}

%\newenvironment{proof}
%               {\hspace{-1\parindent}\textbf{Proof:}}
%               {\hfill $\square$\vspace{1\baselineskip}}

\title{Resource Typing in Guru}

\authorinfo{Aaron Stump}
{CS Dept.\\ The University of Iowa, USA}
{astump@acm.org}

\authorinfo{Evan Austin}
{EECS Dept.\\ The University of Kansas, USA}
{ecaustin@ittc.ku.edu}
%\date{}

\maketitle

%\thispagestyle{empty}

\begin{abstract}
This paper presents a resource typing framework for the Guru
verified-programming language, in which abstractions for various kinds
of program resources can be defined.  Implemented examples include
reference-counted data, mutable arrays, and heap-allocated mutable
aliased data.  The approach enables efficient, type-safe programming
with mutable and aliased data structures, with explicit deallocation
(not garbage collection).  We evaluate performance of the approach
with two verified benchmarks, one involving mutable arrays, and
another involving FIFO queues.
\end{abstract}

\category{D.3.2}{Programming Languages}{Applicative (functional) languages}
\category{F.3.1}{Logics and Meanings of Programs}{Mechanical verification}
\category{F.4.1}{Mathematical Logic}{Mechanical theorem proving}

\terms
Languages, Verification

\keywords
Dependently Typed Programming, Resource Types, Aliasing, Language-Based Verification

\section{Introduction}

Dependent types are of significant current interest for practical
verified programming, as a substantial number of recent works attest
(e.g.,~\cite{oury+08,norell07,pasalic+07,sheard06,nanevski+08,licata+05,chenxi05}).
Dependent types hold out the promise of incrementally extending the
verification power of traditional type systems.  With traditional
verification methods, there is a significant shift in thinking
required for programmers to apply the method.
%For
%example, with traditional approaches based on Hoare Logic, one in
%general needs to annotate loops with loop invariants written in a
%separate logical language, not likely to be familiar to mainstream
%programmers.  
In contrast, with dependent types, programmers can incrementally
enrich types with which they are well familiar.  For example, it is a
relatively small conceptual leap to go from a type like $\langle
\texttt{list}\ A\rangle$ (for homogeneous lists) to a type like
$\langle \texttt{list}\ A\ n\rangle$ (for homogeneous lists of length
$n$).  Of course, the richer the properties expressed by the types,
the more verification burden we might expect.  But still, with
dependent types, there seems a better chance of incrementally
exploring the continuum of correctness, starting with current
programming practice, than with traditional verification methods.

But current dependent type systems are largely confined to functional
programming languages.  Despite the great intellectual depth of the
literature on functional programming, and enthusiastic user
communities, industrial adoption of functional programming is
miniscule compared to languages like C/C++ or Java.  There are, no
doubt, many non-technical reasons for this, such as maturity of
standard libraries (it is very difficult to compete, for example, with
the massive standard library for Java).  But there is at least one
technical issue which has always held back functional programming
languages, and others based on automatic memory management:
efficiency.  Again, despite tremendous invention in the field, garbage
collection remains costly, as several recent studies
report~\cite{Xian2008,hertz+05}.

This paper shows how to combine the expressive power of dependent
types with linear types to define imperative abstractions for
different kinds of resources, without the need for garbage collection.
Combined linear and dependent types are potent verification tools
(cf.~\cite{swamy+09,zhu+05}).  This is carried out in the context of
the Guru verified-programming language, a dependently typed pure
functional programming language.  Resources are created and consumed
explicitly.  The type system rules out memory errors such as accessing
deallocated or uninitialized memory by ensuring that each resource is
consumed exactly once.  The definition of a resource is flexible, so
that resource consumption need not correspond to memory deallocation.
This flexibility is achieved through a \textbf{resource typing
  framework}, in which different resource management schemes can be
defined by a set of primitive types, along with operations on those
types.  The definition of a term-level primitive consists of its type,
an optional pure functional term which models the primitive, and
(imperative) C code which the Guru compiler will use for the
implementation of the primitive.  The connection between this C code and
the pure functional model (if supplied) must be trusted.  The resource
typing framework -- in particular, its linear types -- helps ensure
that the abstraction is used only in ways where the behaviors of the
imperative implementation and functional model coincide.  But since
the number of lines of C code required to implement, say, mutable
arrays is very small (on the order of tens of lines), we contend that
this is a reasonable compromise between trustedness and the burden of
verification.

We begin with an overview of resource typing in \textsc{Guru},
focusing on several example resource types.  We then consider several
example imperative abstractions built using those resource types.
These include abstractions for mutable arrays, and an example of an
abstraction using aliasing, namely FIFO queues.  We give a system of
resource-typing rules, and state its soundness with respect to a
resource-manipulating semantics.  Finally, we give some empirical
results comparing our approach to memory management with that based on
garbage collection in \textsc{OCaml} and \textsc{Haskell} on some
benchmark programs.

The following \textsc{Subversion} command will check out the version
of \textsc{Guru} and the sources for all the examples in this paper,
from the Google Code repository for \textsc{Guru}, at
\url{www.guru-lang.org}.  Most are in \textsc{Guru}'s standard
library, \texttt{guru-lang/lib/}.  A few (noted below) are in a
special directory along with the source of this paper:
\texttt{guru-lang/papers/resource-typing/tests/}:
{\footnotesize
\begin{verbatim}
svn co
 http://guru-lang.googlecode.com/svn/branches/plpv2010/ 
\end{verbatim}
}

\section{Overview of \textsc{Guru}'s Resource Typing}

In this section, we define a resource-typing framework, orthogonal to
\textsc{Guru}'s datatype typing, upon which we can build various
imperative abstractions.  We start with the familiar idea of modeling
resources using \textbf{linear types}.  A resource is considered to be
an entity to which exactly one other entity refers, while it exists in
the program.  So a resource in this sense is intended to be like a
physical resource, such as a bicycle: at most one person can make use
of the bicycle at a time.  This is the first principle of our
resource-typing framework: we must have exactly one reference to each
extant resource at any point during evaluation.  To this we add a
second principle, inspired by examples of physical resource, like the
bicycle example.  For after all, more than one person can use the
bicycle, if one rides on the handle bars and another on the seat.
Resources often consist of \textbf{sub-resources} (like the handle
bars and seat of the bicycle), which may be used independently.  But
while they are in use, the resource of which they are parts is not
available for use as a whole.  This is our second principle: resources
may be divided into sub-resources, but in that case, the whole
resource cannot be used by another entity until use of the
sub-resources is complete.  
%For another everyday example, consider a
%boxed set of Harry Potter books.  The boxed set is a resource, and the
%books are its sub-resources.  Different people can borrow the
%individual books, but no one can borrow the entire boxed set until all
%the books have been returned to it.

Our framework allows us to track the sub-resource relationship
statically.  It also enforces the linearity discipline.  The framework
allows different resource types to be declared, together with
primitive operations on them.  Primitives are declared with an
optional functional model, and then C code providing its (imperative)
implementation.  When such a primitive is declared, its resource type
is declared axiomatically: we do not check that the functional model
satisfies that resource type, but just take it as a given that the C
code's behavior matches that specified by the resource type.  As
stated above, the behavioral equivalence of the C code and the
functional model is not proven, and must be trusted.  It is our
contention that this is a reasonable tradeoff between verifiability
and trustworthiness.

A final note on \textbf{usability} of the framework: while we are
currently programming directly in the language described below, we
envision this serving as a core language for a surface language with
more inference than we have here (see the Conclusion).

\subsection{Quick introduction to \textsc{Guru}}
\label{sec:guru}

In order to move quickly to concrete examples, we survey first the
syntax and informal semantics of terms and types in \textsc{Guru},
including resource-type annotations.  We omit the syntax for proofs
$P$ and formulas $F$, which are separate syntactic categories in
\textsc{Guru}.  The syntax is given in Figure~\ref{fig:terms}, where
we write $\underline{o}$ to mean an optional occurrence of $o$.

\begin{figure}
\[
\begin{array}{lll}
t & ::= & x\ || \\
\ & \ &   c\ ||\\ 
\ & \ &    \texttt{fun}\ x(\underline{i_1}\ \underline{o_1}\ x_1\,:\,A_1)\cdots(\underline{i_n}\ \underline{o_n}\ x_n\,:\, A_n)\\
\ & \ & \ \ \ \ \ \ \  : \underline{o}\ T .\ t\ ||\\
\ & \ &    (t\ X)\ || \\ 
\ & \ &         \texttt{cast}\ t\ \texttt{by}\ P  \ ||\\
\ & \ &  \texttt{abort}\ T\ ||\\ 
\ & \ &  \texttt{impossible}\ P\ T\ ||\\ 
\ & \ &  \texttt{do}\ t_1\ \ldots\ t_n\ \texttt{end}\ ||\\ 
\ & \ &  \texttt{let}\ x\ \Eq\ t\ \texttt{by}\ y\ \texttt{in}\ t' \ ||\ 
\\ 
\ & \ &  \texttt{match}\, \underline{!}\ t\ \texttt{by}\ x\ y \ \texttt{with}\\
\ & \ &  \ \ \ c_1\ \bar{x}_1\ \texttt{=>}\ t_1 |
 \cdots | c_n\ \bar{x}_n\ \texttt{=>}\ t_n\ \texttt{end}\ ||\\
\ & \ &  \texttt{existse\_term}\ P\ t\ || \\
\ & \ &  @\ t \\
\\
\\
X & ::= & t\ ||\ T\ ||\ P
\\
\\
A & ::= & T\ ||\ \texttt{type}\ ||\ F 
\\
\\
T & ::= & x\ ||\ d\ ||\ \bang\ ||\ 
     \texttt{Fun}(\underline{i}\ \underline{o}\ x\,:\,A) . \underline{o}\ T\ ||\ \langle T\ Y\rangle 
\\
\\
Y & ::= & t\ ||\ T
\\
\\
o & ::= & \#\texttt{untracked}\ ||\ \texttt{spec}\ ||\ \#r\ ||\ \#\langle r\ x\rangle
\\
\\
i & ::= & \textnormal{\^{\ }}\ ||\ !\ ||\ 1
\end{array}
\]
\caption{\label{fig:terms} Terms ($t$) and types ($T$); underlined occurrences are optional}
\end{figure}

Terms and types are as reported in our previous work~\cite{guru09},
except for the ownership annotations $o$, with their accompanying
\emph{consumption annotations} $i$.  For terms $t$, we have general
recursive functions introduced with \texttt{fun}, which lists inputs
and their types first, then a ``:'', then the output type, then a
``.'', and then the body of the function.  We support finite failure
via \texttt{abort}.  Operationally, \texttt{impossible} terms behave
like \texttt{abort}, but come with a proof that that point in the code
can never be reached (because the assumptions in scope at that point
are contradictory).  The \texttt{do-end} construct is just a prefix
notation for sequencing of terms.  It is not the monadic \texttt{do}
familiar from \textsc{Haskell}.  We have explicit casts to change the
type of a term $t$ using an equation between types proved by a proof
$P$.  We also have \texttt{let}- and \texttt{match}-terms.  Right
after the \texttt{match} keyword, we may optionally have $!$, which
means that the scrutinee will not be consumed by the match.  The
default is that the scrutinee will be consumed.  These \texttt{let}-
and \texttt{match}-terms bind additional assumption variables after
the \texttt{by} keyword, which provide information in the rest of
those terms about the computation that must have led there.  For
example, in \texttt{let}-terms, the assumption variable has classifier
$\{ x = t \}$, following the \texttt{let}-definition of $x$ to $t$.
This is useful information in the body of the \texttt{let}, since it
tells us that $t$ has terminated, and that $x$ is provably equal to
its value.  The \texttt{existse\_term} is a rarely used construct for
elimination of a proved existential formula, inside a term.  The final
term construct $@\ t$ is for compressing chains of ownership,
discussed further below.

A few words are warranted on the Operational Type Theory (OpTT) upon
which \textsc{Guru} is based.  Provable equality between terms in
\textsc{Guru} has the meaning that the two terms either both diverge
using \textsc{Guru}'s call-by-value operational semantics, or join at
a common value.  Provable equality between types has the meaning that
the types are equal modulo provable equality of any terms they contain
(as indices to indexed datatypes).  OpTT preserves decidability of
typing and logical consistency in the presence of general recursion.
Definitional (automatic) equality is very weak, and does not include
computation.  Casts are used to change the type of a term, thus
placing evaluation under the control of the programmer, instead of the
type checker.  The axiom $\texttt{join}_n\ t_1\ t_2$ states that $t_1$
and $t_2$ are joinable in at most $n$ steps (in practice, $n$ is
omitted and enforced by a global timeout).  To ensure logical
consistency, proofs are syntactically separated from terms, and a
straightforward termination analysis is used to ensure that induction
proofs are well-founded.  The separation of logical and computational
parts in a type-theoretic language has been independently proposed in
Luo's Logic-Enriched Type Theory~\cite{luo08}.

\subsection{Ownership annotations}

Let us now consider the ownership annotations $o$ and consumption
annotations $i$ (of Figure~\ref{fig:terms}).  Every term in
\textsc{Guru} has a resource type, as well as a datatype.  The
ownership annotations $o$ are given for function inputs and outputs,
in order to state that resource type, as well as to express
subresource relationships.  The different kinds of resource types
are:
\begin{itemize}
\item \textbf{\#\texttt{untracked}}.  This is for data that are not
  being treated as resources, such as scalar data (booleans, machine
  words, etc.).  Note that we do not perform closure conversion in
  \textsc{Guru}, so that we may also treat functions as untracked data
  (closure conversion would require tracking the memory for the
  closure).

\item \textbf{\texttt{spec}}. This is for specificational data, that
  are dropped during compilation and by \textsc{Guru}'s definitional
  equality.  Note that the current design does not allow declaring an
  additional resource type (other than \texttt{spec}) for
  specificational data.

\item \textbf{\#\texttt{r}}. This is for data of a declared resource
  type \texttt{r}.  We envision library designers crafting a small
  number of resource types (as we have done so far), that can then be
  used by \textsc{Guru} programmers.
 
\item \textbf{\#\texttt{<r\ x>}}. This is for data of a declared
  resource type \texttt{r}, which are subresources of $x$, where $x$
  is a variable.  If we have $\texttt{y} : \#\texttt{<r\ x>}$, then we say that
  $y$ is pinning $x$.  Our static analysis, defined below, will
  require that all pinning resources of $x$ are consumed before $x$
  itself is consumed.

\end{itemize}

\noindent If an ownership annotation is omitted, \textsc{Guru} chooses
a default of either \#\texttt{untracked}, if the datatype is flat
(like \texttt{bool}), a function type, or declared untracked by the
user; or else \#\texttt{unowned}, discussed below.

Consumption annotations $i$ refine somewhat the basic
resource-tracking of linear types.  Without any consumption
annotation, the meaning is that the input must be consumed exactly
once by the function.  An input can be consumed by passing it to
another function whose type asserts that it will consume the input
(including resource-managing primitive functions which actually
deallocate memory).  This includes passing the input to a term
constructor.  It can also be consumed by returning it from the
function.  The additional annotations $i$ mean:

\begin{itemize}

\item \textbf{\^{\ }}: consume (exactly once) but do not return.  This annotation
  means that the input is to be consumed by the function, but not
  returned, neither directly, nor in another data structure.

\item \textbf{!}: do not consume.  This annotation means that the
  input will not be consumed at all by the function.  This is
  primarily used when defining primitive operations for resource
  types.

\item \textbf{1}: consume, where returning counts as consuming.  This
  annotation is not implemented in \textsc{Guru}, since leaving off
  the consumption annotation already has this meaning.  $1$ is included
  just for uniformity below.

\end{itemize}

\section{Example Resource Types}

In this section, we consider examples of resource types and their
associated primitive operations.  We use these resource types to
help build higher-level imperative abstractions, described later
in the paper.

\subsection{Reference-Counted Data}
\label{sec:unowned}

The most commonly used resource type in \textsc{Guru} is
\#\texttt{unowned}.  This is the resource type for reference-counted
data, which are not pinned by some other reference.
Figure~\ref{fig:unowned} gives a listing from the \texttt{unowned.g}
library file (see \texttt{guru-lang/lib} on the 1.0 branch in the
Google Code repository, via \texttt{www.guru-lang.org}), which
declares this resource type.  This listing has four commands: a
\texttt{ResourceType}-command to declare the \texttt{unowned} resource
type; an \texttt{Init}-command to declare how subdata are to be
initialized when extracting them during pattern-matching; and two
\texttt{Define}-commands, defining primitive functions for
incrementing and decrementing the reference count.  Let us look at
these in turn:

\begin{itemize}
\item the \texttt{ResourceType}-command.  This declares the resource
  type, together with a primitive function \texttt{consume\_unowned}.
  This function will be used by \textsc{Guru} for consuming
  elements of that resource type, in one special situation:
  \texttt{match}-terms.  By default, when we match on a piece of data,
  that data will be consumed.  The \textsc{Guru} compiler will insert
  a call to this \texttt{consume}-function after extracting the
  subdata.

  The functional model given for \texttt{consume\_unowned} is a
  trivial \textsc{Guru} function.  The C code for
  \texttt{consume\_unowned}, which follows ``\texttt{<<END}'', uses
  functions \texttt{dec}, \texttt{op}, and \texttt{release}, provided
  by the \textsc{Guru} runtime, to check the reference counts, which
  are stored in all but the low 8 bits of a single word per
  constructor application.  If the reference count is 0 (so all those
  bits at and above bit 8 are off), then the data is released.  (So
  the \textsc{Guru} runtime provides special built-in support for
  reference counting.)

\item the \texttt{Init}-command.  This specifies how to initialize
  subdata $y$ of $x$ when we pattern-match on $x$.  The \textsc{Guru}
  compiler will insert a call to the given C function, whenever the
  resource type of the scrutinee of a \texttt{match} is the same as
  the first resource type listed (for $x$) in the
  \texttt{Init}-command, and the resource type of the subdatum is the
  same as the second resource type listed (for $y$).  Here, we
  initialize reference-counted subdata by incrementing their reference
  counts.  Consuming a piece of data will consume its subdata, so if
  the reference count of the scrutinee falls to $0$, we will decrement
  the reference counts of its subdata when it is consumed.  That is
  why we increment the subdata's reference counts with this
  \texttt{Init}-function, so that we still have a reference to them
  even after the scrutinee is consumed.  Note that while this use of
  \texttt{Init}-commands could require a number of commands quadratic
  in the number of resource types, we anticipate that the number of
  resource types, particularly used in combination with each other,
  will be small (less than 10, say).
  
\item The \texttt{Define}-commands.  The keyword \texttt{primitive}
signals that these are primitives.  Here, it is the resource types declared for
the primitives that are of most interest.  For \texttt{inc}, the resource type
is
\[
\texttt{Fun(spec A:type)(! \#unowned y:A).\#unowned A}
\]
\noindent This says that \texttt{inc} takes a \texttt{type} $A$, and
then an element $y$ of that type.  The resource type of $y$ is
\#\texttt{unowned}, and the consumption annotation is $!$, indicating
that this $y$ will not be consumed by \texttt{inc}.  Then \texttt{inc}
will produce a new \#\texttt{unowned} $A$ as output.  So the type is
telling us (and the resource-tracking analysis) that when we increment
a reference count of some object $x$, we do not consume the reference
to $x$, but gain an additional new reference to it from \texttt{inc}.
So where there was one reference to $x$ before, there are now two
references.  So we are thinking of each reference to $x$ as a separate
resource to be tracked.

The type for \texttt{dec} just says that it consumes the given
reference.  The consumption annotation \^{\ } says that it does not
return that reference -- but this is not important in this case.

\end{itemize}

The attentive reader might be wondering about why we use \texttt{spec}
with the type $A$ given to \texttt{inc}, but not with the type $A$
given to \texttt{dec}.  The reason is that at runtime, compiled
\textsc{Guru} programs pass integers corresponding to type families
(that is, one integer for all different instances of the type family
for homogeneous lists).  This is for the benefit of the runtime
function \texttt{release}, which actually manages memory for data
constructed with the different term constructors.  So even though all
type annotations are dropped during formal reasoning in
\textsc{Guru}'s logic, some are preserved during compilation.  Here,
it turns out that \texttt{inc} does not need the type at runtime,
while \texttt{dec} (which ends up calling \texttt{release}) does.
Note that the C code for the one case has $A$ as an argument, while
in the other it does not. 

\begin{figure*}
\small
\begin{verbatim}
ResourceType unowned with
  Define primitive consume_unowned : Fun(A:type)(^ #unowned r:A).void 
  := fun(A:type)(r:A).voidi
<<END
  inline void gconsume_unowned(int A, void *r) {
    if (r == 0) return;
    dec(r);
    // fprintf(stdout,"gdec(%x) = %d\n", r, op(r) >> 8);
    if (op(r) < 256)
      release(A,r,1);
  }
END.

Init ginit_unowned_unowned(#unowned x)(#unowned y).#unowned 
<<END
  inline void *ginit_unowned_unowned(int A,void *x,void *y) {
    ginc(y);
    return y;
  }
END.

Define primitive inc : Fun(spec A:type)(! #unowned y:A).#unowned A := fun(A:type)(y:A).y 
<<END
  inline void *ginc(void *y) {
    inc(y);
    // fprintf(stdout,"ginc(%x) = %d\n", y, op(y) >> 8);
    return y;
  }
END.

Define primitive dec : Fun(A:type)(^#unowned y:A).void := fun(A:type)(y:A).voidi 
<<END
  #define gdec(A,y) gconsume_unowned(A,y)
END.
\end{verbatim}
\caption{The \texttt{unowned} resource type}
\label{fig:unowned}
\end{figure*}

\subsection{Owned Data}

One criticism of reference counting as a memory management technique
is that memory traffic generated by incrementing and decrementing
reference counts is costly in terms of performance.  We mitigate this
cost by defining a second resource type, for references which may be
created and consumed without increments and decrements of the
reference count.  The idea is that if such a reference $r$ is
(ultimately) pinning a reference-counted reference $x$ (that is, an
\texttt{unowned} reference), then it is safe to compute with $r$,
without fear that the referenced data object's memory will be
reclaimed.  The reference-counted reference $x$ cannot be consumed
until $r$ is, since $r$ is pinning $x$.  So we can use $r$ safely
without incrementing the reference count.  This idea is the reason for
the \^{\ } consumption annotation.  We need to know that the
pinning reference $r$ is really gone, before it is safe to consume
$x$.  If the pinning reference has been consumed by being returned by
the function, possibly inside a data structure, then this sort of
consumption is not sufficient to guarantee safety: the type system
will have lost track of the connection between $r$ and $x$.  If, on
the other hand, a function is known to consume $r$ and not return it,
then this means that the reference $r$ is dead, and no longer pins
$x$.

We list some representative primitives for this resource type in
Figure~\ref{fig:owned} (for the rest, see
\texttt{guru-lang/lib/owned.g}), and explain them as follows:

\begin{itemize}
\item \textbf{\texttt{consume\_owned}}.  As explained, we do not need to 
decrement the reference count for $x$ when we consume it.

\item \textbf{\texttt{inspect}}. This takes a reference-counted
  (\texttt{unowned}) reference $x$, and returns an \texttt{owned}
  reference pinning $x$.  This is how we obtain an \texttt{owned}
  reference.

\item \textbf{\texttt{clone\_owned}}. This takes an \texttt{owned}
  reference $y$ and creates a new \texttt{owned} reference pinning
  $y$.  Chains of ownership, where $x$ pins $y$ and $y$ pins $z$, can
  be built up in this way.  Occasionally it is necessary to collapse
  such chains, in which case we use the $@\ t$ term-construct
  (Figure~\ref{fig:terms}).  If $x$ pins $y$ and $y$ pins $z$, then
  $@\ x$ pins $z$ directly.

\item \textbf{\texttt{inc\_owned}}. This takes an \texttt{owned}
  reference $y$ and creates a new \texttt{unowned} reference to the
  same object.  This requires, of course, that the reference count
  be incremented.

\item \texttt{ginit\_owned\_unowned}. This initializes an
  \texttt{unowned} subdatum $y$ of an \texttt{owned} scrutinee $x$.
  In this case, the initialized $y$ pins $x$.  This initialization has
  the effect of propagating the \texttt{owned} resource type during
  pattern-matching, and is another point at which chains of ownership
  can develop.

\end{itemize}

\begin{figure*}
\small
\begin{verbatim}
ResourceType owned with 
  Define primitive consume_owned : Fun(A:type)(^#owned x:A).void
  := fun(A:type)(x:A).voidi <<END
  inline void gconsume_owned(int A, void *x) { }
END.

Define primitive inspect : Fun(spec A:type)(!#unowned x:A).#<owned x> A 
  := fun(A:type)(x:A).x <<END
  #define ginspect(x) x
END.

Define primitive clone_owned : Fun(spec A:type)(! #owned y:A).#<owned y> A
  := fun(A:type)(y:A).y <<END
  #define gclone_owned(y) y
END.

Define primitive inc_owned : Fun(spec A:type)(!#owned y:A).#unowned A
  := fun(A:type)(y:A).y <<END
  inline void *ginc_owned(void *y) {
    inc(y);
    return y;
  }
END.

Init ginit_owned_unowned(#owned x)(#unowned y).#<owned x> <<END
  #define ginit_owned_unowned(A,x,y) y
END.
\end{verbatim}
\caption{The \texttt{owned} resource type (selected primitives)}
\label{fig:owned}
\end{figure*}

\subsection{Unique Data}

A simple resource type is that for an object to which exactly one
reference is allowed while the object exists, without the possibility
of multiple references as above (although we will relax that slightly
below).  The resource type for this is \texttt{unique}, defined in
Figure~\ref{fig:unique}.  The interesting point to note here is that
when pattern-matching on a \texttt{unique} object, we are insisting
(with the \texttt{must\_consume\_scrutinee} directive) that the
scrutinee is indeed consumed.  We do not provide any way of
initializing \texttt{unique} subdata from a scrutinee which is not
itself \texttt{unique}.  This ensures that \texttt{unique} must
propagate from subdata to containing data.  That is necessary, of
course, to prevent duplicating a \texttt{unique} resource via a
duplicated reference to an object containing the resource.  If a term
constructor's type does not declare that it creates \texttt{unique}
data, while stating that it accepts \texttt{unique} arguments, then
that constructor could be applied, but never pattern-matched against
(due to the deliberate omission of the appropriate
\texttt{Init}-function).

\begin{figure*}
\small
\begin{verbatim}
ResourceType unique with
  Define primitive consume_unique : Fun(A:type)(^#unique x:A).void
    := fun(A:type)(x:A).voidi <<END
  inline void gconsume_unique(int A, void *x) {
    release(A,x,0);
  }
END.

Init must_consume_scrutinee ginit_unique_unique(#unique x)(#unique y).#unique <<END
  #define ginit_unique_unique(A,x,y) y
END.

Init must_consume_scrutinee ginit_unique_owned(#unique x)(#owned y).#owned <<END
  #define ginit_unique_owned(A,x,y) y
END.

Init must_consume_scrutinee ginit_unique_unowned(#unique x)(#unowned y).#unowned <<END
  #define ginit_unique_unowned(A,x,y) y
END.
\end{verbatim}
\caption{The \texttt{unique} resource type}
\label{fig:unique}
\end{figure*}

\subsection{Unique-owned resource type}

We introduce a final resource type \texttt{unique\_owned}, which we
use to implement readers/writers locking.  We can obtain multiple
\texttt{unique\_owned} references $r$ from a \texttt{unique} reference
$x$, but those references pin $x$.  Our higher-level imperative
abstractions insist that operations which update the object $o$ to
which $x$ refers consume $x$, producing as output a new reference to
the (imperatively) updated $o$.  Since they consume the
\texttt{unique} $x$, our resource-tracking discipline forbids there
from being any extant pinning references to $x$.  This ensures
(statically) that all read-only references to $o$ are consumed, before
the read-write reference $x$ can be used.  The \texttt{Init}-function
for initializing an \texttt{unowned} subdatum $y$ of a
\texttt{unique\_owned} scrutinee $x$ initializes that subdatum to be
of resource type \texttt{<owned x>}, thus propagating the
``\texttt{owned}'' quality.  The listing is omitted for space
reasons, but see \texttt{lib/unique\_owned.g}.

\section{Array Abstractions}

In this section, we consider two different abstractions of mutable
arrays, built using the resource types described in the previous
section.  These array abstractions include a small number of primitive
operations, with a very small number of additional lines of C code to
trust (20 for \texttt{warray}, 27 for \texttt{qcharray}).  

\subsection{Mutable Arrays of Reference-Counted Data}
\label{sec:arrays}

We can easily define an abstraction for mutable word-indexed arrays
storing reference-counted data, where we statically ensure that
array-bounds are respected.  The corresponding library file is
\texttt{warray.g}.  First, we define \texttt{<warray A n>}, where $A$
is a type and $n$ is a \texttt{word}.  The intention is that $n$ is
the length, as a word, of the array.  We define that
\texttt{warray}-type as a primitive abbreviation for \texttt{<vec A
  (to\_nat wordlen n)>}.  Primitive abbreviations like this one are
not included in \textsc{Guru}'s definitional equality when type
checking code that may be compiled.  This is to ensure that compiled
code only accesses \texttt{warray}s using \texttt{warray}'s primitive
operations, not operations (like pattern-matching) on the \texttt{vec}
type.  In code specially marked as specificational, or in proofs,
however, the abbreviation is included in definitional equality, since
for reasoning purposes, we will treat \texttt{warray}s as vectors.

It should also be explained that in that definition, \texttt{word} is
the type for 32-bit machine words, which are functionally modeled as
bitvectors of length 32, and imperatively implemented as actual
machine words.  The \texttt{to\_nat} function converts a bitvector to
a unary natural number, for more convenient reasoning.  Appealing to
such functions inside of types does not cause those functions to be
executed at run-time, or even compile-time.  Since compile-time
evaluation is entirely under the control of the programmer, he may
find ways to prove equalities about applications of (expensive)
functions like \texttt{to\_nat} without actually performing long
executions of them.

Then we define four primitive operations, to allocate, deallocate,
read to, and write from a \texttt{warray}.  Figure~\ref{fig:warrayget}
lists the code for just one representative operation, namely the read
operation.  The functional model of this operation is just to use
\texttt{vec\_get} to look up the value stored at location
\texttt{(to\_nat wordlen i)}.  The running time for this, of course,
is exponential in the size (as a word) of $i$.  The imperative
implementation just uses the standard C constant-time array access
operation.  Note that the input $u$ to the primitive is required to be
a proof that the index $i$ is less than the length $n$ of the array,
when those words are converted to \texttt{nat}s.  By requiring this
proof, we statically ensure array-bounds are respected whenever
\texttt{warray\_get} is used.  The \texttt{warray\_set} primitive has
a similar requirement.  

As far as resource typing goes, we see that \texttt{warray\_get}
requires a \texttt{unique\_owned} \texttt{warray} $l$.  So this is a
read-only reference to the \texttt{warray}.  The function returns an
element of datatype $A$, with resource type \texttt{<owned l>}.  We
use a pinning \texttt{owned} resource type here instead of
\texttt{unowned}, just to avoid incrementing the reference count for
the returned data.  The calling context can always do that, using
\texttt{inc\_owned}.  Of course, the array $l$ is pinned by this
\texttt{<owned l>} reference, until the latter is consumed.  So we
will not be able to update the array until the pinning reference has
been consumed (possibly just by exchanging it for an \texttt{unowned}
reference).

\begin{figure}
\small
\begin{verbatim}
Define primitive warray_get
   : Fun(spec A:type)(spec n:word)
        (! #unique_owned l:<warray A n>)
        (i:word)
        (u:{(lt (to_nat i) (to_nat n)) = tt}).
        #<owned l> A := 
  fun(A:type)(spec n:word)
     (l:<warray A n>)(i:word)
     (u:{(lt (to_nat i) (to_nat n)) = tt}). 
   (vec_get A (to_nat wordlen n) l (to_nat wordlen i) u)
<<END
inline void* gwarray_get(void **l, int i) { return l[i]; }
END.
\end{verbatim}
\caption{Definition of \texttt{warray\_get} primitive}
\label{fig:warrayget}
\end{figure}

\subsection{Mutable Arrays of Unique Data}

We now consider mutable arrays storing unique data.  Currently, we
have implemented this imperative abstraction for character-indexed
arrays, \texttt{qcharray}.  Implementing it for word-indexed arrays
poses no significant challenges, but remains to future work.  The
basic idea here is to track statically the set of indices at which the
array is currently unavailable for reading.  Such indices correspond
either to uninitialized locations in the array, or to locations where
the data has been ``checked out'' for writing.  The set of such
indices is tracked as a \texttt{string} (which in \textsc{Guru} is a
list of characters), included as an additional parameter to the
\texttt{qcharray} type. Checking out the element at index $c$ from a
\texttt{qcharray} (i.e., reading from the array) requires a proof that
$c$ is not in the list of indices which are already checked out.
Similarly, checking in an element at index $c$ from a
\texttt{qcharray} (i.e., writing to the array) requires that the list
of checked-out elements contains $c$.  We list the types of selected
primitives for \texttt{qcharray} in Figure~\ref{fig:qcharray}.  To
discuss just a couple of those: the type for \texttt{qcharray\_empty}
says that it returns a \texttt{qcharray} where all the indices are
checked out (since they are all uninitialized); and
\texttt{qcharray\_out} returns a pair, as an element of the
\texttt{qcharray\_mod\_t} type listed in the figure, of the element
that has been checked out, and the \texttt{qcharray}.  The type of the
\texttt{qcharray} has been modified so that it now includes the index
$c$ for the checked-out element.

\begin{figure}
\small
\begin{verbatim}
qcharray_empty 
: Fun(A:type).#unique <qcharray A all_chars>

Inductive qcharray_mod_t
   : Fun(A:type)(c:char)(s:string).type :=
 mk_qcharray_mod : 
   Fun(A:type)(#unique a:A)
      (spec c:char)(spec s:string)
      (#unique l:<qcharray A (stringc c s)>).
    #unique <qcharray_mod_t A c s>.

qcharray_out 
: Fun(A:type)(#untracked c:char)(spec s : string)
     (#unique l:<qcharray A s>)
     (u : { (string_mem c s) = ff}).
   #unique <qcharray_mod_t A c s>

qcharray_in
: Fun(spec A:type)(#untracked c:char)
     (#unique a:A)(spec s1 s2:string)
     (#unique l:<qcharray A (string_app s1 (stringc c s2))>). 
   #unique <qcharray A (string_app s1 s2)>
\end{verbatim}
\caption{Types for selected \texttt{qcharray} primitives}
\label{fig:qcharray}
\end{figure}

\section{Abstractions Using Aliasing}

In this section, we consider how to implement imperative abstractions
with aliased heap-allocated data.  We first discuss the
\texttt{rheaplet} abstraction, which provides a functional model of a
portion of the heap, and then see how to use it to implement FIFO
queues.  We give first a simple unverified implementation of FIFO
queues, and then a verified version.  In the course of implementing
the verified version, we uncovered a bug in the unverified version,
which was not revealed by the benchmark testcase (used for the
empirical evaluation below).

\subsection{Modeling the aliased heap with \texttt{rheaplet}s}

To implement an aliased data structure, it is necessary to find a way
to model of the non-local communication which arises when a structure
is updated via one reference, and then read via another reference.  We
rely here on a functional model of a portion of the heap, which we
call a \texttt{rheaplet}.  The ``-\texttt{let}'' is to indicate that
this is just a portion of the heap, not its entirety; and the
``\texttt{r}-'' is because we use run-time reference counting of the
number of aliases to each cell, to manage memory.  This approach
statically guarantees that uninitialized or deleted memory will never
be read or written.  It should be acknowledged from the beginning,
however, that this approach can leak memory if cycles are formed in
the reference graph.  We have several things to say about that issue,
in Section~\ref{sec:future} below.

The \texttt{rheaplet} abstraction is based on a type family
\texttt{<rheaplet A I>}, for heaplets storing aliased data of type
$A$.  The $I$ is a \texttt{rheaplet\_id}, which we use to prevent
(statically) aliases associated with one heaplet from being used to
access another.  The type of aliases is then \texttt{<alias I>}, for
aliases into the heaplet with heaplet id $I$.  In the functional
model, a heaplet is a list of elements of type $A$, and an alias is a
unary natural number giving a position into that list.  In the
imperative implementation, \texttt{rheaplet}s and
\texttt{rheaplet\_id}s are just dummy scalar values (retained during
compilation because we currently do not support specificational data
of resource type other than the default, and so these cannot be
declared specificational).  The imperative implementation of aliases
is explained next.

There are three \texttt{rheaplet} primitives.  The first
(``\texttt{rheaplet\_in}'') is for adding some data to the
\texttt{rheaplet}.  In the functional model, the data is added by
appending it to the list which functionally models the heaplet.  In
the imperative implementation, adding data to a heaplet allocates a
3-word cell in memory, with one word each for a reference count for
the number of outstanding aliases, the type (represented as an
integer, as mentioned above) of the data, and the pointer to the data.
The imperative implementation of aliases is just as references to this
3-word cell.  This means that aliases can be managed using the usual
primitives for reference-counted data (Section~\ref{sec:unowned}
above).  The level of indirection added by the 3-word cell means that
we can change the object to which the aliases refer, by changing the
pointer (the last word) in the cell.  This indirection imposes an
additional cost in memory and running time, and so it is particularly
important to evaluate it empirically, as we do below.  The
\texttt{rheaplet\_in} function returns the updated heaplet, and the
new alias.

Finally, the types for the primitives for reading and writing via an
alias are given in Figure~\ref{fig:rheaplet}.  They are similar in
spirit to the types for reading and writing mutable arrays above.  The
main point to note is the use of the \texttt{rheaplet\_id} $I$ to
connect an alias with its \texttt{rheaplet}.  Even though we may
deallocate a cell in the imperative implementation (if the number of
outstanding aliases to that cell falls to zero), in the functional
model we never remove an element from the list which models the
heaplet.  So an alias is always a valid index into the list of its
associated heaplet.

\begin{figure}
\small
\begin{verbatim}
rheaplet_get
: Fun(spec A:type)(spec I:rheaplet_id)(^#owned p:<alias I>)
     (!#unique_owned h:<rheaplet A I>).
   #<owned h> A

rheaplet_set
: Fun(A:type)(spec I:rheaplet_id)(^#owned p:<alias I>)
     (#unique h:<rheaplet A I>)(a:A).
   #unique <rheaplet A I>
\end{verbatim}
\caption{The types of the primitives for reading and writing an \texttt{rheaplet} via an alias}
\label{fig:rheaplet}
\end{figure}

\subsection{Unverified FIFO Queues}

Using the \texttt{rheaplet} primitives, we have implemented a
statically memory-safe FIFO queue abstraction.  The source code for
this may be found as \texttt{queue.g} in the tests subdirectory for
this paper (see the Introduction).  A queue consists of a
singly-linked list from the \texttt{qout}-end, where data is dequeued,
to the \texttt{qin} end, where it is enqueued.  Each cell of the queue
is allocated in a heaplet associated with the queue.  This is because
the last such cell has two aliases (all the others have one), which
ends up forcing all cells to be heaplet-allocated.
Figure~\ref{fig:queue} lists the datatype definition for queues
(omitting a simple definition for queue cells), and then the types for
the operations on queues.  Note that these operations are not
primitives.  They are built using the above abstractions, and require
no additional untrusted C code.  Note also that since this data
structure does not form cycles in the heap, our reference-counting
approach for \texttt{rheaplet}s is sufficient to reclaim all
unreachable memory (though this is not statically verified).

We just consider a few points about the typings in the figure.  Both
empty (\texttt{queue\_datan}) and non-empty (\texttt{queue\_datac})
queues have a unique reference to a \texttt{rheaplet} of queue cells.
Additionally, non-empty queues have references to \texttt{qout} and
\texttt{qin}.  Allocating a new queue with \texttt{queue\_new} (not
shown) creates a new heaplet.  The \texttt{queue\_front} and
\texttt{dequeue} functions require a proof that the queue is
non-empty.  These proofs allow us to show (via
\texttt{impossible}-terms, mentioned in Section~\ref{sec:guru} above)
that several situations cannot arise.  There is only one place where
the queue code uses \texttt{abort}, in response to violation of an
unverified invariant that the \texttt{qin}-cell has no next pointer.
We will eliminate this \texttt{abort} in the next section.  The total
number of lines of code, with comments, is 138.  Of these, only 6 are
proofs, used in the \texttt{impossible}-terms just mentioned.

\begin{figure}
\small
\begin{verbatim}
Define type_family_abbrev
  rheaplet_queue := 
fun(A:type)(I:rheaplet_id).
 <rheaplet <queue_cell A I> I>.

Inductive queue : Fun(A:type).type :=
 queue_datac
  : Fun(A:type)(spec I:rheaplet_id)
       (#unique h:<rheaplet_queue A I>)
       (qin qout : <alias I>).
     #unique <queue A>
| queue_datan
   : Fun(A:type)(spec I:rheaplet_id)
        (#unique h:<rheaplet_queue A I>).
      #unique <queue A>.

queue_is_empty
 : Fun(spec A:type)
      (^#unique_owned q:<queue A>).bool

queue_front
 : Fun(spec A:type)
      (^#unique_owned q:<queue A>)
      (u:{ (queue_is_empty q) = ff }). A 
 
enqueue
 : Fun(spec A:type)
      (#unique q:<queue A>)(a:A).
    #unique <queue A> 

dequeue
 : Fun(spec A:type)
      (#unique q:<queue A>)
      (u:{ (queue_is_empty q) = ff }).
    #unique <queue A>
\end{verbatim}
\caption{Types for the unverified queue abstraction}
\label{fig:queue}
\end{figure}

\subsection{Verified FIFO Queues}
\label{sec:queues}

In order to verify that the \texttt{qin}-cell of a queue has no next
pointer, we must reason explicitly about the heap.  This was not
required for the implementation of the queue data structure as above.
In particular, it is necessary to reason explicitly about the equality
and disequality of aliases.  This imposes a significant burden of
proof, which future work must lighten in order for this method to be
practical; more is said about this in Section~\ref{sec:future} below.
Nevertheless, the current approach provides an adequate foundation for
more automated methods of verifying such properties.

Figure~\ref{fig:queue2} shows the invariants needed to verify that the
\texttt{qin}-cell of the queue has no next pointer, expressed as
additional proof arguments required by the constructors for the
verified queue type.  The full source code for the verified queue may
be found as \texttt{queue2.g} in the tests directory for this paper
(see the Introduction).  The new invariants for non-empty queues are
the following:

\begin{itemize}
\item \textbf{\texttt{inv\_qin}.}  This expresses that \texttt{qin} is
a valid index into the heaplet \texttt{h}; \texttt{inv\_qout} is similar.
\item \textbf{\texttt{inv}.} This expresses that all the next pointers
  in the queue cells stored in the heaplet \texttt{h} are valid
  indices into \texttt{h}.  Note that the \texttt{@<...>} construct is
  \textsc{Guru} notation for application of a defined predicate symbol
  (whose definition here we omit).
\item \textbf{\texttt{inv\_qin2}.} This expresses our desired
  invariant, that the \texttt{qin} cell has no next pointer.
\end{itemize}

\noindent Some discussion of these invariants is warranted.  First,
the reader might well wonder why we are introducing these explicit
statements that aliases are valid indices into the heaplet.  After
all, we maintain the relationship between aliases and heaplets
statically, using the shared \texttt{rheaplet\_id}.  
While we have designed our static typing to ensure that heaplets
are accessed only by their own aliases, this does not imply a principle like:

{\small
\begin{verbatim}
Forall(A:type)(I:rheaplet_id)
      (a:<alias I>)(r:<rheaplet A I>).
  { (lt a (length r)) = tt }
\end{verbatim}
}

\noindent Indeed, this principle is unfortunately false: aliases
created later need not be valid indices into earlier versions of
heaplet with heaplet id $I$.  The problem is due at least in part to
the fact that our logic is intuitionistic (\textsc{Guru}'s design
would also allow classical logic), while the state involved must be
reasoned about linearly.  How to recover something like this principle
is not yet clear, though see the discussion below
(Section~\ref{sec:future}).

The verified versions of the queue functions of Figure~\ref{fig:queue}
above have the same types as given there.  The difference is that now
whenever the constructors for the \texttt{queue} type are used to
(re)construct a queue, we must supply proofs of our invariant
properties.  These proofs are built manually, with modest aid from
\textsc{Guru}'s lone but reasonably powerful tactic, called
\texttt{hypjoin}.  This tactic automatically proves terms equal (under
some reasonable conditions about termination) iff they are joinable in
the operational semantics modulo ground equations specified (by giving
proofs for them) by the user~\cite{petcher09,petcher08}.  Despite this
tactic, the proofs are mostly manual, swelling the original 138 lines
for the unverified queue to 310, not counting general list lemmas from
the standard library and general heaplet lemmas, such as, for example,
this one expressing that the operation of adding an element to the
heaplet is successful (reading the updated heaplet $h'$ using the new
alias $p$ returns the added element $a$):

{\small
\begin{verbatim}
rheaplet_in_get
: Forall(A:type)(I:rheaplet_id)(h h':<rheaplet A I>)
     (p:<alias I>)(a:A)
     (u:{(rheaplet_in h a) = (return_rheaplet_in h' p)}).
   { (rheaplet_get p h') = a }
\end{verbatim}
}
  
As noted above, while proving these invariants for the verified queue
we uncovered a bug in the unverified queue.  The bug is rather
insidious, because it is not uncovered by our benchmark testcase.
This benchmark enqueues space-delimited strings read from standard
input, and then dequeues them all, printing the last string dequeued.
The bug was that the unverified code was inadvertently swapping
\texttt{qin} and \texttt{qout} in the \texttt{dequeue} operation.  On
our benchmark, the effect was to dequeue the first element (at
\texttt{qout}), and then the last element (at the new \texttt{qout},
which is the previous \texttt{qin}).  Since there is no next pointer
from that erroneous \texttt{qout}, it appears to \texttt{dequeue} as
though it has removed the last element.  All memory is correctly
freed, thanks to our linear typing (and confirmed by
\texttt{valgrind}).  The error was detected during proof of this
intensionally quite different property, because the assertion that
\texttt{qin} has no next-pointer cannot be proved when we swap
\texttt{qin} and \texttt{qout}.

\begin{figure}
\begin{verbatim}
Inductive queue : Fun(A:type).type :=
 queue_datac
 : Fun(A:type)(spec I:rheaplet_id)
      (#unique h:<rheaplet_queue A I>)
      (qin qout : <alias I>)
      (inv_qin : {(lt qin (length h)) = tt})
      (inv_qout : {(lt qout (length h)) = tt})
      (inv : @<rheaplet_queue_inv A I h>)
      (inv_qin2 : {(queue_cell_has_next
                     (rheaplet_get qin h)) = ff}).
    #unique <queue A>
| queue_datan
  : Fun(A:type)(spec I:rheaplet_id)
       (#unique h:<rheaplet_queue A I>)
       (inv:@<rheaplet_queue_inv A I h>).
     #unique <queue A>.
\end{verbatim}
\caption{The main datatype, with invariants, for the verified queue abstraction}
\label{fig:queue2}
\end{figure}

\section{Type System}

We now give a formal definition of the resource-tracking static
analysis implemented in \textsc{Guru} as an abstract operational
semantics.  We formulate soundness with respect to an operational
semantics operating on certain (finite) reference graphs.  The
reference graphs are in the form of what we call \emph{shallowly
  augmented forests}: they are forests, except that nodes may have
multiple predecessors in the graph, but then at most one of those
predecessors can itself have a predecessor. This is sufficient for our
functional-modeling approach, since it captures the idea that data are
all tree-structured, except that there may be some additional
references to the nodes (corresponding to predecessors that have no
predecessor).  

\subsection{Assumptions on the terms}

We assume that terms have already passed the datatype checker, and
that calls to initialization functions have been inserted as follows.
(Our implementation inserts these calls during resource-type checking,
where we just check simple typing with respect to the resource types
like \texttt{unique}.)  Suppose we have an \texttt{Init}-command
beginning like this:

\begin{verbatim}
Init ginit_r_rr(#r scrut)(#rr subdat).#rrr
\end{verbatim}

\noindent Suppose further we have a \texttt{match}-case with a
scrutinee of resource type \texttt{r}, and a pattern $c\ \bar{x}$,
where $y\equiv x_i$ is of resource type \texttt{rr}.  Also, assume the
scrutinee is syntactically a variable $x$ (we perform a simple compiler
transformation to ensure this, before resource tracking begins).  Then
the body of the case will begin with nested \texttt{let}-bindings,
including one of the form

\begin{verbatim}
let y = ginit_r_rr(x,y) in
\end{verbatim}

\noindent We further assign the \texttt{ginit\_r\_rr} function this
resource type, which omits the datatype annotations (like
\texttt{<list A>} or similar) and shows only the resource-type (like
\texttt{unowned} or \texttt{unique}) and consumption annotations:

\begin{verbatim}
Fun(! rr scrut)(^ r subdat).rrr
\end{verbatim}

\noindent Also, we assume that uses of \texttt{match} without the
optional $!$ indication (see Figure~\ref{fig:terms}) have been
translated to uses which do include this annotation, and where the
body of each case contains, right after the prefix of
\texttt{let}-bindings just discussed, a call to the appropriate
\texttt{consume}-function for \texttt{r} (the scrutinee's resource
type), on the scrutinee.  This desugars \texttt{match}, which does
consume the scrutinee, right after initialization of subdata, into
\texttt{match!}, which does not.  

We assume the (singly recursive) terms have already been
defunctionalized into global first-order recursive equations of the
forms $(f\ \bar{x}) = t$.  This is a standard step for compilation of
functional languages, and allows us to assume below that all functions
are defined at the top level.  Term constructors we denote with $c$,
and primitives with $p$. Finally, we assume that proofs (including
assumption variables) as well as data marked specificational have
already been dropped, after datatype checking.  Also, we do not treat
\texttt{existse\_term}- and \texttt{do}-terms (both easily handled),
for space reasons.

\subsection{Graph-based operational semantics}

We define a graph-based evaluation relation $\Delta; t \Downarrow_k r;
\Delta'$, where $t$ is a term; $r$ (here and below) is a variable;
$\Delta, \Delta'$ are functions from variables $x$ to values of the
form $(c\ x_1\ \ldots\ x_n)$, with $c$ an arity-$n$ term constructor;
and $k$ a natural number (thus describing inductive data more
explicitly as a graph).  If $\Delta(x) = (c\ \bar{x})$, this means
variable $x$ is instantiated with that term, where the variables
$\bar{x}$ may have their own instantiations in $\Delta$ (similarly for
$\Delta'$).  Furthermore, $k$ is a natural number used as a timer; it
is a technical device to allow us to distinguish diverging and
finitely failing computations (for which we derive a $\perp$ judgment
in the limit) from ones which fail due to resource errors (for which
there is no derivation possible at all using the rules).

Figure~\ref{fig:opsemt} gives the rules.  The first is for expiration
of the timer.  In this case, the form of the judgment has $\perp$ on
the right hand side of the $\Downarrow$, indicating timeout with the
given timer.  We take these $\perp$-judgments to propagate strictly
for all rules (that is, if a premise is provable with such a judgment,
then the rule also concludes with $\perp$) -- but we leave this
machinery unformalized.  The second rule is for evaluation of
variables, the third for evaluation of constructor applications (we
consider uses of 0-ary constructors as degenerate applications), and
the fourth for evaluation of function applications.  The next two
rules, for applications of primitives $p$, consume resources by
modifying the map $\Delta$ according to $p$'s type, and return values
using a functional model unconstrained by that type.  In particular,
this is where we might add an additional shallow reference into a tree.

Consuming resources when a primitive is applied is done using a
meta-theoretic function application $U(\Delta,\bar{x},\bar{i})$,
defined in Figure~\ref{fig:consume}.  We write $p::\bar{i}$ in the
rule to indicate that the consumption annotations $\bar{i}$ are the
ones stated in $p$'s type for its arguments.  We also inductively
define a function ``$-$'' on $\Delta$ and an
$x\in\textit{dom}(\Delta)$ for removing subgraphs from the graph
defined by $\Delta$:
\[
\begin{array}{ll}
\infer{S_0 - x = (\bigcap_i S_i) \setminus \{(x,(c\ \bar{x}))\} }
      {S_0(x) = (c\ \bar{x}) & \forall i.(S_0 - x_i = S_i)} &
\infer{S - x = S}
      {x\not\in\textit{dom}(S)}
\end{array}
\]
\noindent Note that if an inference contains an undefined application
of a meta-theoretic function, we interpret this to mean the inference
is not allowed.  

\begin{figure}
\begin{eqnarray*}
U(\Delta,x,\bar{x},1,\bar{i}) & = & U(\Delta - x ,\bar{x},\bar{i})\textnormal{, if }x\in\textit{dom}(\Delta)\\
U(\Delta,x,\bar{x},!,\bar{i}) & = & U(\Delta ,\bar{x},\bar{i}) \\
U(\Delta,x,\bar{x},\textnormal{\^{\ }},\bar{i}) & = & U(\Delta - x ,\bar{x},\bar{i})\textnormal{, if }x\in\textit{dom}(\Delta)\\
U(\Delta,\cdot,\cdot) & = & \Delta
\end{eqnarray*}
\caption{Meta-theoretic function for consuming $\Delta$ resources}
\label{fig:consume}
\end{figure}

 For simplicity, we assume that if a primitive $p$ lacks a functional
 model, then its return type is an instance of a primitively defined
 type.  In that case, the second rule for applications of primitives
 just returns a new variable, bound to an uninterpreted entity
 ``$\cdot$'' in the heap.  Earlier stages of type-checking rule out
 pattern-matching on elements of primitively defined types, but it is
 more convenient not to rely on that here (so we include the second
 \texttt{match}-rule).  Note that in the \texttt{match}-rules we use a
 more compact meta-theoretic notation for \texttt{match}-terms.
 Finally, if for all $k$, $\Delta; t\Downarrow_k\ \perp$, then we
 write $\Delta; t\Downarrow\ \perp$, and say that $t$ diverges. This
 relation is not recursively enumerable, but that is not problematic
 since it is used only in the formulation of soundness of our
 resource-tracking analysis.

\begin{figure}
\[
\begin{array}{c}
\infer{\Delta; t\Downarrow_0\ \perp}{\ } \\ \\

\infer{\Delta; x\Downarrow_{k+1} x; \Delta}{x\in\textit{dom}(\Delta)} \\ \\

\infer{\Delta_1; (c\ t_1\ldots t_n)\Downarrow_{k+1} r; \{(r,(c\ \bar{r}))\}\cup\Delta_{n+1}}
      {\begin{array}{l}
       \forall i. (\Delta_i; t_i \Downarrow_{k} r_i; \Delta_{i+1}) \\
       r\not\in\textit{dom}(\Delta_{n+1})
       \end{array}} \\ \\

\infer{\Delta_1; (f\ t_1\ldots t_n)\Downarrow_{k+1} r;\Delta'}
      {\begin{array}{l}
       \forall i. (\Delta_i; t_i \Downarrow_{k} r_i; \Delta_{i+1}) \\
       (f\ \bar{x}) = t \\
       \Delta_{n+1}; [\bar{r}/\bar{x}]t \Downarrow_k r; \Delta'
       \end{array}} \\ \\

\infer{\Delta_1; (p\ t_1\ldots t_n)\Downarrow_{k+1} r;\Delta'}
      {\begin{array}{l}
       \forall i. (\Delta_i; t_i \Downarrow_k r_i; \Delta_{i+1}) \\
       (p\ \bar{x}) = t \\ 
       p::\bar{i} \\
       U(\Delta_{n+1},\bar{x},\bar{i}); [\bar{r}/\bar{x}]t \Downarrow_k r; \Delta'
       \end{array}} \\ \\

\infer{\Delta_1; (p\ t_1\ldots t_n)\Downarrow_{k+1} r;\{(r,\cdot)\}\cup U(\Delta_{n+1},\bar{x},\bar{i})}
      {\begin{array}{l}
       \forall i. (\Delta_i; t_i \Downarrow_k r_i; \Delta_{i+1}) \\
       \textnormal{no functional model for }p\\ 
       p::\bar{i} \\
       r\not\in\textit{dom}(\Delta)
       \end{array}} \\ \\

\infer{\Delta; \texttt{match}\ t\ \texttt{with}\ j.(c_j\ \bar{x}_j \texttt{=>} t_j) \Downarrow_{k+1} r;\Delta'}
      {\begin{array}{l}
       \Delta; t \Downarrow_k (c_j\ \bar{r}); \Delta'' \\
       \Delta''; [\bar{r}/\bar{x}]t_j \Downarrow_k r; \Delta'
       \end{array}} \\ \\

\infer{\Delta; \texttt{match}\ t\ \texttt{with}\ j.(c_j\ \bar{x}_j \texttt{=>} t_j) \Downarrow_{k+1}\ \perp}
      {\begin{array}{l}
       \Delta; t \Downarrow_k (d\ \bar{r}); \Delta'' \\
       d\not\in\{c_1,\ldots,c_n\}
       \end{array}} \\ \\

\infer{\Delta; \texttt{let}\ x\ =\ t\ \texttt{in}\ t' \Downarrow_{k+1} r'';\Delta''}
      {\begin{array}{l}
       \Delta; t \Downarrow_{k} r; \Delta' \\
       \Delta'; [r/x]t' \Downarrow_k r''; \Delta''
       \end{array}} \\ \\

\infer{\Delta; \texttt{abort} \Downarrow_k\ \perp}{\ } \\ \\

\infer{\Delta; @\ t \Downarrow_{k+1} r; \Delta'}
      {\Delta; t\Downarrow_k r; \Delta'}

\end{array}
\]
\caption{Graph-based operational semantics}
\label{fig:opsemt}
\end{figure}

\subsection{Resource-tracking analysis}

Next, we define an abstract, modular evaluation relation $\Theta; t
\Downarrow x; \Theta'$, where $\Theta, \Theta'$ are functions from
variables $x$ to a consumption annotation $i$ (see
Figure~\ref{fig:terms}) or an expression $\langle y \rangle$, where
$y$ is a variable pinned by $x$.  The rules are given in
Figure~\ref{fig:opsema}, using a meta-theoretic function $V$ similar
to $U$ above, and defined in Figure~\ref{fig:consumea}.  (The first
rule of that Figure removes a resource $x$ from $\Theta$ if (1) it is
there, (2) not a pinning reference (these must be removed with
consumption annotation \^{\ }), and (3) not itself pinned by another
reference.)  The rules of Figure~\ref{fig:opsema} are syntax-directed
and all decrease the size of the term from conclusion to premises.
Hence, in a standard way, they determine a terminating algorithm.  The
first two rules are for variables and constructor applications, while
in the third, $g$ is for a regular function symbol or a primitive
function symbol: the two cases are handled the same way here, based on
their resource type.  Note that our indexed notation in the second and
third rule forces the initial $\Theta$ ($\Theta_1$) to propagate
sequentially through the evaluation of the arguments, ending with a
final $\Theta_{n+1}$. We treat \texttt{abort}, similarly here as
above, with a judgment form ending in $\perp$, which is propagated via
straightforward omitted rules.

The rules are modular, in the sense that they rely on types for all
the regularly defined functions they call.  This is justified by our
assumption that our code has already been defunctionalized into
top-level definitions.  These types (given in the original source
code) are confirmed using the following rule:
\[
\infer{f::\bar{i}}
      {\begin{array}{l}
       (f\ \bar{x}) = t\\
       \textnormal{Assuming }f::\bar{i}\textnormal{, we have: }\{ (x_1,i_1), \ldots, (x_n,i_n)\}; t \Downarrow r; \{(r,1)\}
       \end{array}}
\]    

The definition of the $V$ function is the heart of the
resource-tracking analysis.  It insists that pinning references cannot
be consumed by being returned; that is, the corresponding consumption
annotation must be \^{\ }.  It also insists that
references cannot be consumed until all their pinning references are.

Some simplifying design choices have been made in this algorithm.  For
example, the rule for constructor applications disallows arguments
marked with \^{\ } (meaning to be consumed but not returned).  Also,
branches in \texttt{match}-cases must have the same effect on
$\Theta$, and may not return pinning references.  Note that having the
same effect on $\Theta$ is not the same as having exactly the same
pattern of resource allocations and consumptions.  One
\texttt{match}-case might create a reference, which another does not.
But then that first \texttt{match}-case must ensure that the reference
is consumed by the end of the case, so that the observable effect on
$\Theta$ is the same as in the other case.  These restrictions could
perhaps be relaxed.  Also, we have not formalized consuming resources
named in (global) definitions, although our implementation supports
this.

\begin{figure}
\[
\begin{array}{c}
\infer{\Theta; x\Downarrow x; \Theta}{x\in\textit{dom}(\Theta)} \\ \\

\infer{\Theta_1; (c\ t_1\ldots t_n)\Downarrow r;\{(r,1)\}\cup V(\Theta_{n+1},\bar{r},\bar{i})}
      {\begin{array}{l}
       \forall j. (\Theta_j; t_j \Downarrow r_j; \Theta_{j+1}) \\
       c::\bar{i} \\
       \textnormal{\^{\ }}\not\in \{ \bar{i} \} \\
       r\not\in\textit{dom}(\Theta)
       \end{array}} \\ \\

\infer{\Theta_1; (g\ t_1\ldots t_n)\Downarrow r; \{(r,1)\}\cup V(\Theta_{n+1},\bar{r},\bar{i})}
      {\begin{array}{l}
       \forall j. (\Theta_j; t_j \Downarrow r_j; \Theta_{j+1}) \\
       g::\bar{i} \\
       r\not\in\textit{dom}(\Theta)
       \end{array}} \\ \\

\infer{\Theta; \texttt{match}\ x\ \texttt{with}\ j.(c_j\ \bar{x}_j \texttt{=>} t_j) \Downarrow r;\{(r,i)\}\cup\Theta'}
      {\begin{array}{l}
       \forall j.(\Theta, \bar{x}_j; t_j \Downarrow r_j; \{(r_j,i)\}\cup \Theta')\\
       r\not\in\textit{dom}(\Theta')
       \end{array}} \\ \\

\infer{\Theta; \texttt{let}\ x\ =\ t\ \texttt{in}\ t' \Downarrow r'';\Theta''}
      {\begin{array}{l}
       \Theta; t \Downarrow r; \Theta' \\
       \Theta'; [r/x]t' \Downarrow r''; \Theta''
       \end{array}} \\ \\

\infer{\Theta; \texttt{abort} \Downarrow\ \perp}{\ } \\ \\

\infer{\Theta; @\ t \Downarrow r ; \{(r,\langle z\rangle)\}\cup \Theta' }
      {\begin{array}{l}
       \Theta; t \Downarrow r; \{(r,\langle y\rangle)\}\cup\Theta' \\
       (y,\langle z\rangle)\in \Theta'
       \end{array}}

\end{array}
\]
\caption{Resource-tracking abstract operational semantics}
\label{fig:opsema}
\end{figure}

\begin{figure}
\begin{eqnarray*}
V(\Theta,x,\bar{x},1,\bar{i}) & = & V(\Theta - x ,\bar{x},\bar{i})\textnormal{, if }
\begin{array}{l}
1.\  x\in\textit{dom}(\Theta) \\ 
2.\  \forall y. \Theta(x) \neq \langle y \rangle \\ 
3.\  \forall (y,\langle z \rangle)\in\Theta,\ z\neq x \\
\end{array} \\
V(\Theta,x,\bar{x},!,\bar{i}) & = & V(\Theta ,\bar{x},\bar{i}) \\
V(\Theta,x,\bar{x},\textnormal{\^{\ }},\bar{i}) & = & V(\Theta - x ,\bar{x},\bar{i})\textnormal{, if }x\in\textit{dom}(\Theta)\\
V(\Theta,\cdot,\cdot) & = & \Theta
\end{eqnarray*}
\caption{Meta-theoretic function for consuming $\Theta$ resources}
\label{fig:consumea}
\end{figure}

 Let us define $\Delta^*(t)$ for shallowly augmented forests
$\Delta$ to be the term which is just like $t$ except that we replace
each $x\in\textit{dom}(\Delta)$ with $\Delta^*(\Delta(x))$.  The fact
that $\Delta$ is a shallowly augmented forest ensures well-foundedness
of this definition.  If for every $x\in\textit{dom}(\Delta)$ we have
that $\Delta^*(x)$ is ground (no free variables), then we will call
$\Delta$ ground.  Also, let us write $\sigma::\Theta\to\Delta$ to indicate
that:
\begin{enumerate}
\item $\sigma$ is an injective function from $\textit{dom}(\Theta)$ to $\textit{dom}(\Delta)$, and
\item $\forall (x,\langle y \rangle)\in\Theta$:
\begin{enumerate}
\item $y\in\textit{dom}(\Theta)$, and
\item $\Delta^*(\sigma(x))\textnormal{ is a subterm of }\Delta^*(\sigma(y))$ (with both defined).
\end{enumerate}
\end{enumerate}

\begin{proposition}[Type Soundness]
If $\Theta; t \Downarrow r; \Theta'$, then for all ground shallowly
augmented forests $\Delta$ and $\sigma::\Theta\to\Delta$, the
following are both true.
\begin{enumerate}
\item Either $\Delta; \Delta^*(t) \Downarrow\ \perp$, or else $\Delta;
  \Delta^*(t)\Downarrow_k r'; \Delta'$ for some $k$, $r'$, $\sigma'$, and
  $\Delta'$ with:
\begin{enumerate}
\item $\sigma'::\Theta'\to\Delta'$, and
\item $\sigma'(r) = r'$,
\item $\sigma'|_{\textit{dom}(\sigma)} \subset \sigma$.
\end{enumerate}
\item If $(x,\textnormal{\^{\ }})\in\Theta$, then for all
  $y\in\textit{dom}(\Delta)$, $\Delta'^*(\sigma'(x))$ is a
  subterm of $\Delta'^*(y)$ only if it (already) is of $\Delta^*(z)$,
  for some $z\in\textit{dom}(\Delta)$ where $\Delta^*(y)$ is a subterm
  of $\Delta^*(z)$.
\item If $(x,!)\in\Theta$, then $\sigma'(x)$ is defined.
\end{enumerate}
\end{proposition}

\noindent The first condition states that resources are consumed from
$\Delta$ in a way consistent with $\Theta$ (as mediated by $\sigma$).
The second states that evaluation using $\Delta$ cannot cause a
reference marked with \^{\ } (``consume but do not return'') to be
embedded in a value in $\Delta'$, unless it was already so embedded in
$\Delta$.  The third states that references marked not to be consumed
are in fact still present in the final state.  Writing out in detail
the detailed proof of this theorem, which should proceed by induction
on the structure of the assumed typing derivation, with sub-inductions
for relating results of $U$ and $V$, remains to future work.  We may
then easily apply this theorem to conclude that our modular analysis
of recursive functions is sound.

\section{Empirical Results}
\label{sec:emp}

The goal of the following tests is to compare the performance of
mutable arrays and FIFO queues implemented in \textsc{Guru} with
analogous data structures implemented in two of the leading functional
programming languages currently in use, \textsc{Haskell} and
\textsc{OCaml}.  Comparisons are also made to \textsc{Haskell} and
\textsc{OCaml} versions with garbage collection minimized by picking
large enough values for heap sizes so that collection is not necessary
(see the Makefile in the \texttt{tests} directory mentioned in the
Introduction).  The specifications of the test system are described in
Figure \ref{teststats}, including compilers and optimization options.


\begin{figure}
\begin{tabular}{ll}
\texttt{Processor:} & 2.67 GHz Intel Xeon W3520 \\
\texttt{Memory:} & 8 GB \\
\texttt{OS:} & Linux version 2.6.18-164.2.1.el5\\
\texttt{Compilers:} & GHC 6.10.4 -O \\
\ & OCamlOpt 3.11.1 \\
\ & Gnu gcc 4.3.3 -O4
\end{tabular}
\caption{Specifications of the test system.}
\label{teststats}
\end{figure}

\subsection{Mutable Array Test}

The first test uses a verified version of a textbook binary search
algorithm using word-indexed mutable arrays.  What makes the
implementation in \textsc{Guru} unique is that it uses the
\texttt{warray\_get} primitive method mentioned earlier in this paper.
Recall that this method requires a proof that the index of the array
that is to be accessed is less than the length of the array.
Therefore, for binary search to succeed, we must derive a proof that
the middle point to search is less than the length of the array.  This
proof is derived relatively easily given three other proofs: that the
first index of the search space is less than the length of the array,
that the last index of the search space is less than the length of the
array, and that the first index of the search space is less than or
equal to the last index of the search space.  This gives our binary
search function the type shown in Figure \ref{bs}.  Also of note is
the use of a generic comparator function, allowing the algorithm to
work over all types of word-indexed arrays.  Supplying this comparator
function to the search function is similar to defining and utilizing
an instance of the \texttt{Ord} type class for a data type in
\textsc{Haskell}.

\begin{figure}
\begin{verbatim}
Define warray_binary_search
  : Fun(A:type)                                   
       (spec n:word)                              
       (^ #unique_owned l:<warray A n>)           
       (first last:word)                          
       (value:A)                                  
       (c:Fun(^ #owned a b:A). comp)              
       (u:{(lt (to_nat first) (to_nat n)) = tt})  
       (v:{(lt (to_nat last) (to_nat n)) = tt})
       (w:{(le (to_nat first) (to_nat last)) = tt})
       . bool                                     
\end{verbatim}
\caption{The type of the binary search algorithm for word-indexed arrays.}
\label{bs}
\end{figure}

To test the performance of binary search, an ordered array of size
$2^{20}$ is created and then each possible index is searched for.  The
total time to complete both of these tasks is shown in
Figure~\ref{warraystats}.  The different sizes of the binary
executables are somewhat striking, so we include these as well.  The
\textsc{Guru} implementation beats the implementations in
\textsc{Haskell} and \textsc{OCaml}, with garbage collectors on.  The
reports of others, cited in the Introduction, about the performance
penalty of garbage collection are confirmed here.  

\begin{figure}
\begin{center}
\begin{tabular}{| l | l | l |}
\hline
\multicolumn{3}{|c|}{Mutable Array Test} \\
\hline
Language & Avg Real Time & Size of Binary \\
\hline
\textsc{Haskell} & 1.18 s & 581K \\
\textsc{Haskell} (No GC) & 0.49 s & \  \\
\textsc{OCaml} & 0.61 s & 131K \\
\textsc{OCaml} (No GC) & 0.54 s &\  \\
\textsc{Guru} & 0.42 s & 37K \\
\hline
\end{tabular}
\end{center}
\caption{Average real time performance of word-indexed array test.}
\label{warraystats}
\end{figure}

\subsection{FIFO Queue Test}

The second test involves the methods of the FIFO queue described
earlier in the paper.  To begin with, two new FIFO queues are created.
Then the contents of \textit{War and Peace} are enqueued to the first
queue word by word.  The first queue is then dequeued one element at a
time, enqueueing the item into the second queue.  Once the second
queue is full the contents are again dequeued and the last element is
printed to the console.  The results of this test are shown in
Figure~\ref{queuestats}.  We see similar relative performance as for
the previous example.  For this benchmark, \textsc{Guru} is easily
beaten by \textsc{OCaml} and \textsc{Haskell}, if their garbage
collectors are turned off.  This is as one would expect, given the
much greater maturity of their implementations.  Note that the
verified and unverified queues have indistinguishable performance (so
we show just the results for the verified queue).  We conjecture that
branch prediction in the CPU masks the difference between the two
implementations (the verified one does not need to do a run-time check
to detect the situation which is proved to be impossible).

\begin{figure}
\begin{center}
\begin{tabular}{| l | l | l |}
\hline
\multicolumn{3}{|c|}{Queue Test} \\
\hline
Language & Avg Real Time & Size of Binary\\
\hline
\textsc{Haskell} & 1.08 s & 614K \\
\textsc{Haskell} (No GC) & 0.53 s & \  \\
\textsc{OCaml} & 0.66 s & 132K \\
\textsc{OCaml} (No GC) & 0.37 s &\ \\
\textsc{Guru} & 0.60 s & 37K \\
\hline
\end{tabular}
\end{center}
\caption{Average real time performance of queue test.}
\label{queuestats}
\end{figure}

%%TODO explain results; clean up wording; clean up formatting;
\section{Discussion}
\label{sec:future}

There are several points arising in this study that deserve further discussion:

\ 

\textbf{The status of memory leaks.} Our \texttt{rheaplet} abstraction
does not guarantee absence of memory leaks, and indeed, cyclic
structures can be leaked.  Statically ensuring that memory leaks do
not occur appears costly.  Indeed, dynamic methods such as garbage
collection are currently so widely used for just that reason: they
ensure that memory is not leaked, with no additional verification
cost.  But there is a performance cost, which we are trying to avoid.
One could imagine adding garbage-collected regions as additional
resource types.  Indeed, our \texttt{rheaplet} data structure already
compromises performance in order to reduce the burden of static
verification.  If static verification of the absence of memory leaks
is desired, then the approach we would advocate is to use a version of
heaplets (not yet implemented) where the number of outstanding aliases
is statically tracked in the heaplet's type.  

%Verifying that the
%number of aliases is zero when the heaplet is consumed would require
%significant verification effort in general, since one would need to
%maintain invariants expressing the aliasing structure of the heaplet,
%in order to be able to show statically that all aliases have been
%consumed.

%For cyclic structures, it might be necessary to disassemble the
%structure carefully on deallocation, in a reverse manner to its
%construction.  This choreography will be too intricate for the
%verification budget of many applications.

%A final note on memory leaks is that to the best of our knowledge,
%there are no general methods for statically verifying the absence of
%live leaks.  And of course, the absence of leaks of unreachable memory
%is simply an approximation to the truly desired property, which is
%that memory which will no longer be used will be reclaimed.  Dynamic
%leak-pruning methods have been considered (see, e.g.,~\cite{bond+09}),
%but a general theory for statically guaranteeing the absence of live
%leaks where possible appears to be lacking from the literature.

\textbf{On automating proofs.} Our current burden of proof for our
example using aliased heap-allocated data is similar to that of the
2008 version of the Ynot system, at least in being heavier than
desirable~\cite{nanevski+08}.  The 2009 version of Ynot greatly
decreases the burden of manual proof through clever use of automated
tactics (in the \textsc{Coq} proof assistant)~\cite{ynot09}.  We also
agree that automation must be applied to help reduce the manual effort
for proofs about the heap.  Our emphasis in \textsc{Guru} is on
developing predictable and complete tactics like \texttt{hypjoin}.  In
contrast, \textsc{Coq}'s standard tactics include many of a heuristic
and unpredictable nature, which can easily break when a proof script
is modified.

%But the methodology for automation we are
%pursuing in \textsc{Guru} differs greatly from that of \textsc{Coq},
%and of at least some of the tactics reported in~\cite{ynot09}.  The
%\texttt{hypjoin} tactic mentioned above is sound, but also complete
%(under modest conditions about termination of the functional terms
%involved): if two terms are joinable modulo the given ground
%equations, without case splitting or induction, then \texttt{hypjoin}
%will determine that they are.  It is our hypothesis that usability of
%powerful tactics is greatly increased by increased predictability.
%Heuristic methods, or methods lacking a simple independent delineation
%of the class of problems they can solve, seem likely to increase
%fragility of proofs and difficulties for users (particularly less
%proficient ones).  It remains to be seen which tactics fitting such a
%description can be devised for reasoning about the heap, or more
%generally.  Certainly one path is to try to extend methods like
%\texttt{hypjoin} to the non-ground case, but this is likely to be
%technically challenging.

\textbf{The provable connection between aliases and heaplets.} The use
of heaplets to model just a portion of the heap is similar in spirit
to the use of separating conjunction in the separation logic approach
used by the Ynot authors~\cite{nanevski+08,reynolds02}.  It also
distinguishes such approaches from previous work of Zhu and Xi on
stateful views, where a type $t@l$ is used to express that an element
of type $t$ occurs at address $l$, in the single implicit global
heap~\cite{zhu+05}.  Above we were forced to reason in more detail
than we would like about the fact that aliases associated statically
(via their \texttt{rheaplet\_id}) with a heaplet are actually valid
indices into that heaplet (considered as a list).  This must be
improved in future work.


%It might be
%possible to reduce this burden of proof by adding some more direct
%language support for heaplets.  For example, we could add a feature to
%add proofs automatically inside terms -- via new assumption variables
%-- that valid indices remain valid under all the heaplet operations
%(recalling that in the functional model, we never remove a value from
%a heaplet).  This would greatly reduce the burden of proof for the
%verified queue example, which is mostly concerned with proving that
%aliases are valid heap indices.

\section{Conclusion}

We have seen how a resource typing framework based on two simple
principles can support a variety of low- and higher-level imperative
abstractions:
\begin{itemize}
\item Resources may be used by at most one party at a time.
\item Resources may be divided into subresources, which must be
  returned before the resource as a whole is usable again.
\end{itemize}
\noindent Future work includes completing the detailed proof of type
soundness, using the formulation devised above. On the practical side,
our focus will be on reducing the burden of annotation and proof in
the ways discussed in the previous section, and also by finding ways
to eliminate the need for programmers to apply resource primitives
like \texttt{inc}, \texttt{dec}, and \texttt{inspect} explicitly.
Ideas from other uniqueness-type systems could help there
(e.g.~\cite{vries08}). A simple first approach might be to devise an
algorithm that will automatically insert calls to these primitives,
given types for all function input variables.

\textbf{Acknowledgements:} the U.S. NSF for support under award
CCF-0841554; and the growing group of students and researchers
experimenting with \textsc{Guru}, especially University of Iowa
students Derek Bruce, Frank Fu, John Hughes, Tianyi Liang, Duckki Oe,
Andrew Reynolds, and Greg Witt.

\bibliographystyle{plainnat}

%\nocite{SH80}
\bibliography{partiality,misc_logic,automated_reasoning,formal_methods,verification,lf,general,refinement,coop_dec_procs,cl,rewriting,theorem_provers,sat,program_analysis,software_engineering,specification,pl,stanford_group,hoas,semantic_programming,misc}


\end{document}
