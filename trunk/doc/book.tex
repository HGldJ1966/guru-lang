\documentclass{book}[12pt]
\usepackage{times}
\usepackage{url}
\usepackage{fullpage}

\newcommand{\guru}[0]{\textsc{Guru}}

\begin{document}

\title{Verified Programming in \guru}

\author{Aaron Stump \\
Computer Science \\
The University of Iowa \\
Iowa City, Iowa, USA
}

\maketitle

\tableofcontents

\chapter{Introduction}
\label{ch1}


\chapter{Functional Programming in \guru}
\label{ch:prog}

Mainstream programming languages like Java and C++, while powerful and
effective for many applications, pose problems for program
verification.  This is for several reasons.  First, these are large
languages, with many different features.  They also come with large
standard libraries, which have to be accounted for in order to verify
programs that use them.  Also, they are based on programming paradigms
for which practically effective formal reasoning principles are still
being worked out.  For example, reasoning about programs even with
such a familiar and seemingly simple feature as \emph{mutable state}
is not at all trivial.  Mutable state means that the value stored in a
variable can be changed later.  The reader perhaps has never even
dreamed there could be languages where this is not the case (where
once a variable is assigned a value, that value cannot be changed).
We will study such a language in this chapter.  Object-orientation of
programs creates additional difficulties for formal reasoning.

Where object-oriented languages are designed around the idea of an
object, functional programming languages are designed around the idea
of a function.  Modern examples with significant user communities and
tool support include Caml (pronounced ``camel'',
http://caml.inria.fr/www) and Haskell (http://www.haskell.org/).
Haskell is particularly interesting for our purposes, because the
language is \emph{pure}: there is no mutable state of any kind.
Indeed, Haskell programs have a remarkable property: any expression in
a program is guaranteed to evaluate in exactly the same way every time
it is evaluated.  This property fails magnificently in mainstream
languages, where expressions like ``\texttt{gettimeofday()}'' are, of
course, intended to evaluate differently each time they are
called.  Reasoning about impure programs requires reasoning about the
state they depend on.  Reasoning about pure programs does not, and is
thus simpler.  

Inside Coq is a pure functional programming language.  The language
is similar in some ways to Caml and Haskell.  In this chapter, we
will learn this functional programming language.  In the next chapter,
we will see how to verify programs written in it.

\section{Inductive Datatypes}

At the heart of functional programming languages like Caml and Haskell
(but not LISP dialects like Scheme) are user-declared inductive
datatypes.  An inductive datatype consists of data which are
incrementally and uniquely built up using a finite set of operations,
called the \emph{constructors} of the datatype.  Incrementally built
up means that bigger data are obtained by gradual augmentation from
smaller data.  Uniquely means that the same piece of data cannot be
built up in two different ways.  Let us consider several examples.

\subsection{Unary Natural Numbers}

The natural numbers are the numbers $0,1,2,\ldots$.  We typically
write numbers in decimal notation.  Unary notation is much simpler.
Essentially, a number like 5 is represented by making 5 marks, for
example like this:

\[ |\ |\ |\ |\ | \]

\noindent A few questions arise.  How do we represent zero?  By zero
marks?  It is then hard to tell if we have written zero or just not
written anything at all.  So, let us write zero as $0$.  Also, how
does this fit the pattern of an inductive datatype?  That is, how are
bigger pieces of data (i.e., bigger numbers) obtained incrementally
and uniquely from smaller ones?  One answer is that a number like five
can be viewed as built up from its \emph{predecessor} 4 by the
\emph{successor} operation, which we will write $S$.  The successor
operation just adds one to a natural number.  In this book, we will
follow Coq in writing the application of a function $f$ to an input
argument $x$ as $f\ x$ or $(f\ x)$.  This is in contrast to other
common mathematical notation, that writes $f(x)$ for function
application.  So the five-fold application of the successor operation
to zero, representing the number 5, is written this way:

\[ (S\ (S\ (S\ (S\ (S\ 0))))) \]

Every natural number is either $0$ or can be built from $0$ by
applying the successor operation a finite number of times.
Furthermore, every natural number is uniquely built that way.  This
would not be true if in addition to $0$ and $S$, we included an
operation $P$ for predecessor.  In that case, there would be an
infinite number of ways to build every number.  For example, $0$ could
be built using just $0$, or also in these ways (and others):

\[ 
\begin{array}{l}
(S\ (P\ 0)) \\
(S\ (S\ (P\ (P\ 0)))) \\
(S\ (S\ (S\ (P\ (P\ (P\ 0)))))) \\
\ldots
\end{array}
\]

\noindent The operations $0$ and $S$ are the \emph{constructors} of
the natural number datatype.

The simplicity of unary natural numbers comes at a price.  The
representation of a number in unary is much larger -- exponentially
larger, in fact -- than its representation in decimal notation.  For
example, it takes very many slash marks or applications of $S$ to
write $100$ (decimal notation) in unary.  In contrast, it only takes 3
digits in decimal.

\subsection{Unary Natural Numbers in Coq}
\label{sec:natcoq}

Coq's standard library includes a definition of unary natural numbers,
and definitions of standard arithmetic functions operating on them.
To play with these, start up the Coq IDE (``Integrated Development
Environment'').  In the left panel, beneath where it says
``*Unnamed Buffer*'', you can enter Coq input.  To evaluate the next
piece of input you have written (once you write some), you can select
\textbf{Navigation - Forward} from the menu.  Alternatively, you can
click on the down triangle on the menu bar.

In Coq, you can get the type of any expression $e$ by evaluating the
Vernacular command ``\texttt{Check e.}'' (including the final period).
(Recall from the Introduction that Vernacular is the top-level command
language, which instructs Coq to perform operations like declaring
datatypes and functions, or, as here, checking types.)  Evaluate the
command:

\begin{verbatim}
Check 3.
\end{verbatim}

\noindent This will cause Coq to print ``\texttt{3 : nat}'' in the bottom
right panel, which Coq uses for displaying output of commands.  Now try

\begin{verbatim}
Check S (S (S (S (S 0)))).
\end{verbatim}

\noindent This causes Coq to print ``\texttt{5 : nat}'' in the output
panel.  Coq prints $5$ here because it knows that $5$ is the preferred
(because more readable) notation for the number we write in unary as
$S\ (S\ (S\ (S\ (S\ 0))))$.  To see the number in unary notation, first
evaluate

\begin{verbatim}
Unset Printing Notations.
\end{verbatim}

\noindent and then evaluate the \texttt{Check} command to compute the
type for $S\ (S\ (S\ (S\ (S\ 0))))$ again.  Now Coq's output panel
will show the number in unary.  If you look carefully, you will see
that it prints $O$ (the capital letter) instead of $0$ (the number).
This is because in Coq's datatype \texttt{nat} for unary natural
numbers, the constructors are declared to be $S$ and $O$ (not $0$).
The number $0$ is just notation for the constructor $O$.  To turn on
printing notations again, you can either evaluate the command

\begin{verbatim}
Set Printing Notations.
\end{verbatim}

\noindent or else you can back up in the Coq window, to ``unevaluate''
the command which turned off the printing of notations.  To back up,
you can select \textbf{Navigation - Forward} from the menu, or click
on the up triangle on the menu bar.
 
You can display the actual declaration of the \texttt{nat} datatype
by evaluating

\begin{verbatim}
Print nat.
\end{verbatim}

\noindent If you do this, you will see the following output in the
output panel:

\begin{verbatim}
Inductive nat : Set :=  O : nat | S : nat -> nat
For S: Argument scope is [nat_scope]
\end{verbatim}

\noindent We will ignore the second line of output here, which
concerns Coq's notation system.  The first line gives the actual
datatype definition.  It begins with the keyword \texttt{Inductive},
and then gives the name of the datatype which is being declared
(\texttt{nat}).  After this comes ``\texttt{: Set}''.  We will not
consider the role of this now, except to say that it tells Coq that
\texttt{nat} is a computational type (Coq has non-computational types,
which are related to proofs).  Then comes ``\texttt{:=}'', which is
followed by declarations for the two constructors, $O$ and $S$, of the
\texttt{nat} datatype.  There are two declarations, separated by a
``\texttt{|}''':

\begin{verbatim}
O : nat 
S : nat -> nat
\end{verbatim}

\noindent The first states that $O$ is a \texttt{nat}.  The second
states that $S$ is a function taking \texttt{nat}s as input and
returning\texttt{nat}s as outputs.  The type for functions is written
with ``\texttt{->}'' in Coq, with the type for input written before
the arrow, and the type for output written after.  You can confirm all
the typings for \texttt{nat} using the Vernacular \texttt{Check}
command.  For example, if you evaluate

\begin{verbatim}
Check nat.
\end{verbatim}

\noindent Coq will report that \texttt{nat} is a \texttt{Set} (that
is, \texttt{nat} is a computational type), just as it was declared
above.

\subsection{Lists}

Computer Science frequently uses lists to hold collections of data.
Lists can also be viewed as constituting an inductive datatype.  There
are two ways to build a list, on this view: we either build an empty
list, holding no data, or else we add a piece of data to an existing
list.  For historical reasons, the empty list is written \texttt{nil}.
To add a piece of data $x$ to the front of an existing list $L$, we
can write $x :: L$.  For example, the list of the first 3 natural numbers
in order from biggest to smallest may be written

\begin{verbatim}
2 :: 1 :: 0 :: nil
\end{verbatim}

\noindent This list is built by putting $2$ onto the front of the list
``\texttt{1 :: 0 :: nil}''.  That list, of course, is built by putting
$1$ onto the front of the list ``\texttt{0 :: nil}'', which itself is
formed by putting $0$ onto the front of the empty list.

Several points about lists should be noted.  First, as an inductive
datatype, lists are somewhat similar to natural numbers.  The empty
list is a bit like $0$, and the operation of putting an element onto
the front of a list is a bit like the successor operation.  Second, it
is quite natural to consider different kinds of lists, as determined
by the kinds of elements they can store.  We can have lists of natural
numbers, as we have just seen.  But we can also have lists of lists of
natural numbers, or lists of other kinds of data.  Most programming
languages require that no one list hold more than one type of
elements.  If a single list may hold multiple types of data, the list
is sometimes called \emph{heterogeneous}.  We will not be much
concerned with heterogeneous lists (although unlike most other
languages, it is possible to define heterogeneous lists in Coq).  If,
however, a single datatype of lists supports different kinds of lists
based on the kinds of elements, than that list type is called
\emph{polymorphic}.

\subsection{Lists in Coq}

Coq's standard library defines polymorphic lists and several
operations on lists.  Unlike the natural numbers, the datatype for
lists is not loaded by default when Coq starts up.  To load it,
evaluate the command

\begin{verbatim}
Require Import List.
\end{verbatim}

\noindent Now try computing the types of a few different lists by
evaluating the following commands (as explained at the start of
Section~\ref{sec:natcoq} above):

\begin{verbatim}
Check 3 :: 4 :: nil.
Check (3 :: 4 :: nil) :: (5 :: 6 :: nil) :: nil.
Check S :: nil.
\end{verbatim}

\noindent The first list has type ``\texttt{list nat}'' in Coq.  The
type ``\texttt{list nat}'' shows that the type of elements of the list
are \texttt{nat}s.  The second list has type ``\texttt{list (list
nat)}'', since each of its elements is a list of \texttt{nat}s.  The
third list has type ``\texttt{list (nat -> nat)}'', because its single
element, $S$, is a function from \texttt{nat}s to \texttt{nat}s (as
discussed in Section~\ref{sec:natcoq}).

\subsubsection{The Declaration of \texttt{list}}

As for natural numbers, you can see the declaration for the list
datatype by evaluating the command:

\begin{verbatim}
Print list.
\end{verbatim}

\noindent If you do this, you will see the following in the output
panel (at the lower right of the Coq window), followed by some
other lines which we will ignore for now:

\begin{verbatim}
Inductive list (A : Set) : Set :=
    nil : list A | cons : A -> list A -> list A.
\end{verbatim}

\noindent This declares the inductive datatype \texttt{list}.  We have
some additional complexity here beyond what we saw for the
\texttt{nat} type, because we are supporting polymorphic lists.  The
notation \texttt{list (A : Set) : Set} in the first line states that
for any computational type ``\texttt{A}'', we have the inductive
datatype ``\texttt{list A}'' for lists of elements of type $A$.  The
two constructors are \texttt{nil} for the empty list, and
\texttt{cons}.  Expressions like \texttt{x::L} are just notation for
\texttt{cons x L}, as you can confirm by turning off the printing of
notations (as in Section~\ref{sec:natcoq} above).  Also, with printing
notations on, if you check the type of \texttt{cons 3 (cons 4 nil)}, for
example, you will see that Coq views this as \texttt{3::4::nil} (of
type \texttt{list nat}).

If you check the type of \texttt{list} with the \texttt{Check}
command, Coq will report that \texttt{list} has type \texttt{Set ->
Set}.  What does this mean?  Since for any computational type $A$, we
have a computational type \texttt{list A}, Coq views \texttt{list} as
a function from types to types.  The input to \texttt{list} is a type
(namely, the type $A$ for elements of the list), and the output is
also a type (the type for lists of elements of type $A$).  To
reinforce your understanding of this idea, try to anticipate what Coq
will print for the following:

\begin{verbatim}
Check list nat.
Check list (list nat).
Check list list.
\end{verbatim}

\noindent The last example does not type check in Coq (and Coq prints
an error in the output panel), because \texttt{list} expects a
computational type as input.  Instead, we have given it a function
from types to types (namely, \texttt{list}).

\subsubsection{The Type of \texttt{cons}}

The constructor \texttt{cons} has a rather unusual type, which you
can see using the \texttt{Check} command:

\begin{verbatim}
cons : forall A : Set, A -> list A -> list A
\end{verbatim}

\noindent What does this mean?  For any type $A$, \texttt{cons} will
take an element $x$ of type $A$, and a list $L$ of $A$s, and build you
the bigger list $x\, ::\, L$.  Coq views \texttt{cons} as taking in
the type $A$ as its first argument, and then accepting $x$ and $L$.
The ``\texttt{forall}'' notation is used here just to give the name
$A$ to that first argument, so it can be used in the subsequent types
(like \texttt{list A}).  One puzzling thing the reader may have
noticed: above, we did not include a type as the first argument to
\texttt{cons}.  For example, we wrote

\begin{verbatim}
cons 3 (cons 4 nil)
\end{verbatim}

\noindent not the following more verbose expression, with the type $A$
of elements included as the first argument to \texttt{cons}:

\begin{verbatim}
cons nat 3 (cons nat 4 nil)
\end{verbatim}

\noindent The reason we can write the shorter form is that Coq's
notation system allows this type argument to be suppressed, since it
can be recovered from the types of the other arguments.  Indeed, if
you look again at the output of the Vernacular command ``\texttt{Print
list.}'', you will see some lines below the datatype declaration,
which mention ``implicit arguments''.  To specify the value for
implicit argument $A$, you can use the notation

\begin{verbatim}
cons(A:=nat)
\end{verbatim}

\noindent If you check the type of this expression, Coq will report
that its type is 

\begin{verbatim}
nat -> list nat -> list nat
\end{verbatim}

Implicit arguments also explain the otherwise rather cryptic error
message you will get if you evaluate the command

\begin{verbatim}
Check nil.
\end{verbatim}

\noindent The problem \texttt{nil}'s type argument is implicit. It can
be recovered only if \texttt{nil} is used in a larger list, where the
types of the elements determine the type \texttt{nil} must have.  When
\texttt{nil} is used just by itself, as here, there is no way to
recover the type argument.  The solution is to use the notation
\texttt{nil(A:=nat)}, for example, to specify the implicit argument.

\subsection{Exercises}

Here are some exercises intended both to help you explore Coq and to
improve your understanding of datatypes.  These exercises, while not
too hard, may require some educated guesses to figure out how to
interact with Coq.

\begin{enumerate}
\item Using just the \texttt{Check} command, see if you can find one
negative consequence of representing numbers in unary (hint: try
checking the types of some numbers).  Can you explain why Coq behaves
the way it does in this case?
\item Coq has a pre-defined type for boolean values, just like other
programming languages.  Try to figure out what this type is and what
its constructors are.
\item An \emph{enumerated} inductive datatype is one with a non-zero,
finite number of elements.  It has a finite number of constructors,
all of which create an element of the type without requiring an
element of the type.  So, the constructors are like $0$ for \texttt{nat}
or \texttt{nil} for \texttt{list}: they do not take in an element of
the same datatype as input, but rather just create an element out of
thin air (or some other data).  The enumerated datatype of primary colors
has three constructors, \texttt{red}, \texttt{green}, and \texttt{blue}.
Define this datatype in Coq.  Once that is working, add constructors for
\texttt{light} and \texttt{dark} (enabling colors like light red or light
blue).  Check the types of some elements of this datatype.
\item The datatype of polymorphic binary trees with data at the nodes
and leaves has two constructors: \texttt{leaf}, which takes a piece of
data and builds a tree holding just that piece of data; and
\texttt{node}, which takes a piece of data and two trees (called the
left and right subtrees), and builds the tree which holds the given
piece of data at the root, and then branches to the given left and
right subtrees.  Following the model for polymorphic lists, define
this datatype in Coq, and check the types of some elements of it.
\end{enumerate}

\subsection{Running Programs in Coq}

We will see in the next section how to write our own recursive
programs in Coq.  For now, we will experiment with some predefined
programs.  First, Coq's standard library includes some basic
arithmetic operations.  To run these programs, you use the Vernacular
\texttt{Eval} command.  For example, to watch Coq add 2 and 2, you can
evaluate the following command:

\begin{verbatim}
Eval compute in 2 + 2.
\end{verbatim}

\noindent Here, the ``+'' function is written using \emph{infix}
notation (meaning, it is written in between its arguments).  You can
also call the plus function using prefix notation (meaning, with the
function written before its arguments), as follows:

\begin{verbatim}
Eval compute in plus 2 2.
\end{verbatim}

\noindent The ``2 + 2'' is just notation (defined using Coq's notation
system) for \texttt{plus 2 2}.

\noindent For another set of examples, Coq's library provides definitions
of several standard operations on lists.  For example, try the following:

\begin{verbatim}
Eval compute in app (1::2::3::nil) (4::5::6::nil).
\end{verbatim}

\noindent What is this program \texttt{app} doing?  Infix notation for
\texttt{app} is ``\texttt{++}''.  Try to repeat the above example
using this notation.

\subsection{More Exercises}

\begin{enumerate}
\item Using \texttt{Check} and \texttt{Eval}, try to figure out some
other arithmetic functions Coq's standard library includes.  Try to
guess their prefix forms.

\item Using \texttt{Check} and \texttt{Eval}, try to figure out what
the following list functions do.  Compute a few simple examples with
them:
\begin{enumerate}
\item \texttt{length}
\item \texttt{nth}
\item \texttt{rev}
\end{enumerate}
\item The list function \texttt{map} is used for applying a function
to every element of a list.  For example

\begin{verbatim}
Eval compute in map S (1::2::3::nil)
\end{verbatim}

\noindent will apply the successor function (\texttt{S}) to every element
of the given list.  What will be the resulting list?  Can you find a way
to use \texttt{map} to add 10 to every element of a list (hint: what is
the type of \texttt{plus 3})?
\end{enumerate}

\subsection{The Whiz Programming Language}
\label{sec:whiz}

Programs can also be viewed as an inductive datatype.  For simplicity,
we will take a rather minimalistic programming language I am calling
\emph{Whiz} for our example, but the same idea would work for more
realistic languages.  Our language is \emph{imperative}, which means
that unlike Coq, it does have mutable state: we can change the values
assigned to variables.  The only values our language supports are
natural numbers.  There are only three ways to change the value of a
variable: incrementing, decrementing, or copying the value of another
variable.  Decrementing 0 returns 0, and all variables are initialized
to 0 before the program begins.  Multiple commands are sequenced using
a semi-colon.  We can also repeat a command while the value of a
specified variable is not 0 (the language's name is inspired by
``while-not-zero'').  An example program is given in
Figure~\ref{fig:whizprog}, where we are using indentation to indicate
the grouping of commands.  The command \texttt{copy x tmp} stores the
value of variable \texttt{x} as the new value of variable
\texttt{tmp}.  Can you figure out what value $z$ will have when this
program terminates?  Can you figure out what this program is doing?

\begin{figure}
\begin{verbatim}
increment x;
increment x;
increment x;
increment y;
increment y;
while-not-zero y 
  copy x tmp;
  while-not-zero tmp
     increment z;
     decrement tmp;
  decrement y
\end{verbatim}
\caption{\label{fig:whizprog}Sample Program in Whiz}
\end{figure}

Programs in Whiz can be viewed as constituting an inductive datatype
as follows.  The smallest programs are increment, decrement, and copy
commands.  So we have constructors for those kinds of programs, just
the way we had a constructor $0$ for the smallest natural number and a
constructor \texttt{nil} for the smallest list.  The increment and
decrement constructors take the name of the variable as their only
argument.  The constructor for the copy command takes the names of two
variables (it will copy the value of the first into the second).
There are two other constructors for programs.  One constructor is for
sequencing commands.  It takes two programs $p$ and $q$ as arguments,
and builds the program $p; q$.  The other constructor is for while
loops.  It takes a variable $v$ and a program $p$ as arguments, and
builds the program $\textit{while-not-zero}\ v\ p$.  So all told, we
have five constructors for programs: \textit{increment},
\textit{decrement}, \textit{copy}, \textit{sequence}, and
\textit{while-not-zero}.

\subsection{The Whiz Programming Language in Coq}

We can define an inductive datatype for Whiz programs in Coq.  There
is one somewhat annoying issue we must deal with first, which is in
representing the variables of Whiz programs.  Coq does not have a
string class, so we cannot represent variables like $x$, $y$, and $z$
from our example program above as ``\texttt{x}'', ``\texttt{y}'', and
``\texttt{z}''.  The easiest way to represent variables in Coq is just
by natural numbers.  We can think of $x$ as variable $0$, $y$ as
variable $1$, $z$ as variable $2$, and \textit{tmp} as variable $3$.
Programs become a little less readable.  Here is the program above
with this representation of its variables:

\begin{verbatim}
increment 0;
increment 0;
increment 0;
increment 1;
increment 1;
while-not-zero 1
  copy 0 3;
  while-not-zero 3
     increment 2;
     decrement 3;
  decrement 1
\end{verbatim}

\noindent The key is to remember that all numbers appearing in Whiz
programs are actually the names of variables.

\begin{figure}
\begin{verbatim}
Inductive whiz : Set :=
  inc : nat -> whiz
| dec : nat -> whiz
| cp : nat -> nat -> whiz
| seq : whiz -> whiz -> whiz
| whilenz : nat -> whiz -> whiz.
\end{verbatim}
\caption{\label{fig:whiz}Whiz Programs as a Coq Datatype}
\end{figure}

Now we can use the declaration in Figure~\ref{fig:whiz} to declare the
inductive datatype of Whiz programs in Coq.  We have constructors for
each construct of Whiz, for example \texttt{inc} for
\texttt{increment}.  A Whiz program has type \texttt{whiz}.  The
syntax we have here is not too bad, except that putting programs in
sequence with \texttt{seq} is much uglier than with semi-colon.  For
example, the Whiz program

\begin{verbatim}
increment 0;
increment 1
\end{verbatim}

\noindent must be written as

\begin{verbatim}
seq (inc 0) (inc 1)
\end{verbatim}

\noindent While we could use Coq's notation system to try to clean
this up, we will content ourselves for the moment with this notation.
It can be made somewhat more manageable with indentation.  For example,
to increment variable 0 up to 4, we can write the following:

\begin{verbatim}
 seq (inc 0)
(seq (inc 0)
(seq (inc 0)
     (inc 0)))
\end{verbatim}

\noindent To experiment with Whiz in Coq, first evaluate the following
command to load the file defining Whiz:

\begin{verbatim}
Require Import "~cse240/whiz".
\end{verbatim}

\noindent You can now print the Whiz datatype with

\begin{verbatim}
Print whiz.
\end{verbatim}

\noindent The sample program is defined to be ``\texttt{prog}'', which
you can also view using \texttt{Print}:

\begin{verbatim}
Print prog.
\end{verbatim}

\noindent The file defining Whiz also includes an evaluator called
\texttt{eval} for executing Whiz programs.  The basic idea is that
you can call the evaluator with a program and a variable as arguments,
and the evaluator will run the program and print the value of the
variable in the final state the program reaches.  For example,
calling \texttt{eval} on the program 

\begin{verbatim}
seq (inc 0) (inc 0)
\end{verbatim}

\noindent with the variable 0 will return 2.  There is one slight
complication here, which brings up a fundamental point about
programming in Coq.  For deep reasons, Coq's type theory requires that
every program terminates on all inputs.  We are not allowed to write a
program that loops forever or could loop forever on some inputs.  It
is very easy to write an infinitely looping Whiz program.  So, how can
we write the Whiz program evaluator \texttt{eval} in such a way that
it always terminates?  The answer is, we have to use a bit of a hack.
We must write \texttt{eval} so that you give it a counter telling how
many steps to execute a program.  If the program is not done running
but the counter has reached 0, evaluation of the program just ends
abruptly.  It turns out to be simpler not to count exactly the actual
number of commands executed, but rather an upper bound of that number
of steps.  (In more detail, the counter is decremented for every
instruction, except that it is duplicated at \texttt{seq}
instructions.) So here is an example of evaluating a Whiz program:

\begin{verbatim}
Eval compute in eval 2 (seq (inc 0) (inc 0)) 0.
\end{verbatim}

\noindent Coq will report that this evaluates to 2, since the program
increments variable $0$ twice, and we then print the result.  The
first argument to \texttt{eval} is the counter.  So we will execute
this program roughly 2 steps.  Increasing the counter will not change
the outcome, since the program really does finish executing with a
counter of 2.  

\subsection{More Exercises}

\begin{enumerate}
\item Write a program that increments variable 12 up to 5.  Figure out
the smallest counter needed to get this program to evaluate to 5 (when
\texttt{eval} is called with variable 12).  What value does \texttt{eval}
report for variables other than 12?

\item Can you write a program that loops forever?  What happens as
you run it and increase its counter?

\item Use \texttt{eval} to confirm (or figure out) what the sample
program given in Section~\ref{sec:whiz} above does (recall that this
program is already typed in for you in the Coq Whiz file, and is
available as ``\texttt{prog}'').  What is the smallest value of the
counter that you need to get the answer to stabilize?

\item Try to write a Whiz program implementing subtraction.

\item (More challenging) Can you write a Whiz program implementing
division (hint: you can do division by repeated subtraction)?

\item (More challenging) The \texttt{eval} function uses a helper
function called \texttt{evalh}.  One of the arguments to this function
is a function from \texttt{nat}s to \texttt{nat}s.  This function is
used to implement the state of the program.  It records the value
currently stored for each variable.  This fits with its type, since
each variable is represented as a \texttt{nat}, and each value stored
in a variable is a \texttt{nat}.  So having a function from
\texttt{nat}s (variables) to \texttt{nat}s (values) lets you look up
the value for a variable just by calling the function.  Try calling
\texttt{evalh} yourself.  You can create your own states using
\texttt{start} and \texttt{update} (first figure out what these do by
using \texttt{Check} and \texttt{Eval}).  It can actually be a lot
easier to understand what is going on if you use

\begin{verbatim}
Eval simpl in 
\end{verbatim}

\noindent instead of

\begin{verbatim}
Eval compute in 
\end{verbatim}

\noindent The former keeps the resulting output more symbolic and
readable, at least in this case, than the latter.

\end{enumerate}

\section{Functional Programs}

As explained at the start of this chapter, we are studying pure
functional programs operating on inductively defined data.  Such
programs have no \emph{side effects}: they cannot change the values of
variables, nor can they perform input or output of any kind while
running.  We will further concern ourselves with programs which can be
easily shown to terminate on all input arguments.  That is, it will
not be possible to write programs which run forever on some inputs.
This requirement of termination is certainly not appropriate for many
settings: for example, operating systems are not designed to terminate
(although unfortunately they sometimes do, due to software bugs!).
Nevertheless, even programs designed to run forever usually contain
many subprograms which are intended to be terminating.  So the study
of terminating programs is relevant even in such settings.

This section explains the three central features used for pure
functional programming with inductive data: anonymous functions,
pattern matching, and recursion.  We will study these features as they
appear in Coq.  Similar constructs appear in languages like Haskell
and Caml, with slightly different syntax.  We start with basic syntax
for function definitions.

\subsection{Function Definitions}

In Coq, we can write function definitions using the \texttt{Definition}
command, like this:

\begin{verbatim}
Definition square(x : nat) : nat := x * x.
\end{verbatim}

\noindent This command tells Coq to add a definition of the
\texttt{square} function, which squares a natural number \texttt{x}.
Let us walk through the notation used in this definition.  First,
there is the \texttt{Definition} keyword.  Next comes the name of the
defined function, \texttt{square}.  The single input \texttt{x} to
\texttt{square} is then listed next in parentheses, together with its
type (\texttt{nat}).  After this, we have ``\texttt{: nat}'', which
declares that the return type of the function will be \texttt{nat}.
It is often possible to omit the declaration of the return type, since
Coq can often infer it.  Then comes ``\texttt{:=}'', and then the
\emph{body} of the function.  The body of the function shows what the
function will return when called.  Here, we are returning
\texttt{x*x}, on input \texttt{x}.

\subsection{Anonymous Functions}

In functional programming languages, functions are first-class data:
they may be passed as inputs to other functions, or returned from
functions.  They may also be created anonymously, and used in code
wherever a function is required.  Anonymous functions are not defined
with a \texttt{Definition} command.  We can construct an anonymous
function for squaring and check its type as follows:

\begin{verbatim}
Check fun (x : nat) => x * x.
\end{verbatim}

\noindent This notation for an anonymous function begins with the
keyword \texttt{fun}.  Then arguments are listed, using the same
syntax as we used for the definition of \texttt{square} above.  No
return type for the function is required, and indeed, it is not
allowed to be listed.  The ``\texttt{=>}'' marks the beginning of the
body of the function, which is \texttt{x*x}.  The type of this function
is $\texttt{nat}\,\texttt{->}\,\texttt{nat}$, as we would expect.

The \texttt{Definition} we gave above for \texttt{square} is actually
treated by Coq internally as the following definition involving an
anonymous function:

\begin{verbatim}
Definition square := fun (x:nat) => x * x.
\end{verbatim}

\noindent You can verify that this is the way Coq views our first
\texttt{Definition}, by printing (with \texttt{Print}) \texttt{square}
as first defined.  You can watch anonymous functions run by applying
them to arguments.  For example, try the following (note that here the
type for the input argument \texttt{x} is omitted, again because Coq
can infer that type from the surrounding context):

\begin{verbatim}
Eval compute in (fun x => x * x) 3.
\end{verbatim}

\subsubsection{Higher-order Functions}

A \emph{higher-order} function is one which accepts a function as an
input.  A relatively simple example is the \texttt{compose} function.
The \emph{composition} of two functions $f$ and $g$ is the function
which on input $x$ returns $f\ (g\ x)$.  Let us define composition for
the special case when $f$ and $g$ both take a \texttt{nat} as input
and return a \texttt{nat} as output.  There are several ways to write
this in Coq, one of which is:

\begin{verbatim}
Definition compose(f : nat -> nat)(g : nat -> nat) := 
  fun x:nat => f (g x).
\end{verbatim}

\noindent Here, we take in the two arguments \texttt{f} and
\texttt{g}, and return their composition.  Coq supports a variety of
abbreviations that are useful here.  For example, we can write
\texttt{compose} more compactly by combining the two argument
declarations:

\begin{verbatim}
Definition compose(f g : nat -> nat) := 
  fun x:nat => f (g x).
\end{verbatim}

The type of \texttt{compose} is then

\begin{verbatim}
(nat -> nat) -> (nat -> nat) -> (nat -> nat)
\end{verbatim}

\noindent This type says that \texttt{compose} takes in two functions
from \texttt{nat} to \texttt{nat} as input, and returns another such
function as output.  Coq will actually print the type of \texttt{compose}
as 

\begin{verbatim}
(nat -> nat) -> (nat -> nat) -> nat -> nat
\end{verbatim}

\noindent reflecting the view that \texttt{compose} takes in three
inputs, $f$, $g$, and $x$, and returns $f\ (g\ x)$.  There is no
essential difference between this view and the view of
\texttt{compose} as accepting $f$ and $g$ and returning a function,
because in Coq, as in other functional languages, functions may be
\emph{partially applied}, to just some of their arguments (first in
the argument list).  If \texttt{compose} is viewed as taking three
arguments, we may nevertheless apply it to just the first two.  This
results in a function, namely one waiting for the third argument.
You can confirm this by checking the type of an expression like
\texttt{compose S S}.

\subsubsection{Polymorphic Functions}

To operate on polymorphic data like polymorphic lists and trees, we
need polymorphic functions.  Polymorphic functions can also be used to
give more generic forms of functions like \texttt{compose}.  It is
perfectly reasonable to define \texttt{compose} as taking in $f$ of
type $B \to C$ and $g$ of type $A \to B$, and returning the
composition of type $A \to C$; where $A$, $B$, and $C$ are any types.
To write such a function in Coq, we just take those three types as
additional arguments, like this (the spacing and indendation are just
for readability):

\begin{verbatim}
Definition compose2(A B C : Set)
                   (f : A -> B)(g : B -> C) := 
  fun x:A => f (g x).
\end{verbatim}

\noindent To use this function, we have to list the three types involved:

\begin{verbatim}
Check compose2 nat nat nat S S.
\end{verbatim}

\noindent This expression has type
$\texttt{nat}\,\texttt{->}\,\texttt{nat}$.  Since it quickly becomes
tiresome to write all these type arguments, we can tell Coq that we
wish to keep them implicit, using the following command:

\begin{verbatim}
Implicit Arguments compose2[A B C].
\end{verbatim}

\noindent We may now write just

\begin{verbatim}
Check compose2 S S.
\end{verbatim}

\noindent Note that sometimes there is not enough information for Coq
to infer the value of an implicit argument.  For example, Coq will report
an error if we evaluate the following command:

\begin{verbatim}
Check compose2 S.
\end{verbatim}

\noindent Since the first function is given as $S$, Coq knows that
the type $B$ is \texttt{nat} and the type $C$ is \texttt{nat}.  But
it does not have enough information to infer what the type $A$ has to
be, and so it reports an error.

\subsection{Pattern Matching}

To inspect inductively defined data, functional programming languages
like Coq's use pattern matching.  A pattern matching expression lists
a term to be pattern matched, followed by a number of cases, separated
by bars (``$|$'').  Each case consists of a pattern and a body.  Here
is a simple example.  We can define the predecessor function on
natural numbers as follows, where we take the predecessor of zero to
be zero (and the predecessor of a non-zero number like $5$ to be $4$,
the greatest smaller number):

\begin{verbatim}
Definition pred(x:nat) := 
  match x with
    0 => 0
  | S y => y
  end.
\end{verbatim}

\noindent The \texttt{match} expression used in the body of
\texttt{pred} is performing a pattern match on \texttt{x}.  If
\texttt{x} is $0$, we return $0$.  That is what the first case (the
line ``\texttt{0}\,\texttt{=>}\,\texttt{0}'') of the \texttt{match}
states.  If \texttt{x} is not $0$, we try the next case.  The pattern
is \texttt{S y}.  The variable \texttt{y} is a \emph{pattern
variable}, which will become associated with the matching part of the
data \texttt{x}.  Suppose the input \texttt{x} to \texttt{pred} is
$2$.  As we know from a previous section, $2$ is really an
abbreviation in Coq for \texttt{S\ (S\ O)}.  This expression does
match the pattern \texttt{S\ y}, if we make \texttt{y} be \texttt{S\
0}.  The body (``\texttt{y}'') of the second case is then returned.  On
input $2$, this will be \texttt{S\ O}, or $1$, as expected.  The end
of the pattern matching expression is indicated using the keyword
\texttt{end}.

While other functional languages allow pattern matching expressions to
leave some cases uncovered, Coq does not.  So Coq will report an error
on the following attempted definition of the function \texttt{tl} to
return $L$ from a list $x::L$ (note that below, \texttt{l'} is not
special syntax: single quotes are allowed to appear in variable
names in Coq):

\begin{verbatim}
Definition tl(A:Set)(l:list A) := match l with
  x::l' => l'
end.
\end{verbatim}

\noindent One solution to this problem is to define \texttt{tl} on
the empty list, for example as follows:

\begin{verbatim}
Definition tl(A:Set)(l:list A) := match l with
  nil => nil
| x::l' => l'
end.
\end{verbatim}

\subsection{Recursion}
\label{sec:rec}

Functional programming gains its full computational power from the
ability to define functions by recursion.  In programming languages
like Java and C++, and even in functional languages like Haskell and
Caml, it is quite easy to write recursive programs which fail to
terminate on some inputs.  Coq is unusual in disallowing this: all
programs are required to terminate on all inputs.  There are different
ways to enforce termination on all inputs.  In Coq, this is done by
requiring that recursive programs have a designated argument called
the \emph{parameter of recursion}, which can obviously be seen to
decrease in size at each recursive call.  Since this argument gets
smaller and smaller every time the function calls itself, eventually
the argument will be of the smallest possible size, and the nested
recursion will stop.

We can see an example of this kind of terminating recursion by
examining the code for addition (using ``\texttt{Print plus.}''):

\begin{verbatim}
plus = 
fix plus (n m : nat) {struct n} : nat :=
  match n with
  | O => m
  | S p => S (plus p m)
  end
\end{verbatim}

\noindent This says that \texttt{plus} is defined to be an anonymous
recursive function.  Anonymous recursive functions are written with
the keyword \texttt{fix}.  Then comes a name to use for making
recursive calls.  Here, as is usually done, that name is informatively
chosen to be \texttt{plus}.  Then comes a declaration for two
arguments, \texttt{n} and \texttt{m}, of type \texttt{nat}.  Then
comes ``\texttt{\{struct n\}}''.  This directive tells Coq that
\texttt{n} is the parameter of recursion.  If there is only one
argument to the recursive function, this directive can be omitted.
The return type of \texttt{nat} is then declared using ``\texttt{:
nat}''.  The body of the function comes next.  In the body, we may
make recursive calls using the name \texttt{plus}, which was listed
right after the \texttt{fix} keyword.  It is required that the
parameter of recursion is \emph{structurally} smaller (hence the
``\texttt{struct}'' directive) at such recursive calls.  Here, we make
a recursive call with \texttt{p} as the first argument (which is the
parameter of recursion).  This \texttt{p} is structurally smaller than
the input argument \texttt{n}, because it is extracted from \texttt{n}
using pattern matching.  It is really a smaller piece of \texttt{n}.

It is common in Coq to write recursive definitions with the slightly
less cumbersome syntax of Figure~\ref{fig:fixplus}.  This is treated
internally by Coq as a definition of \texttt{plus} in terms of an
anonymous recursive function.

\begin{figure}
\begin{verbatim}
Fixpoint plus (n m:nat) {struct n} : nat :=
  match n with
  | O => m
  | S p => S (plus p m)
  end.
\end{verbatim}
\caption{\label{fig:fixplus}Plus Defined Using The Fixpoint Command}
\end{figure}


\subsection{Exercises}

\begin{enumerate}
\item Write an anonymous function which takes a natural number
\texttt{x} and returns the list containing just \texttt{x}.  For
example, if the input is \texttt{3}, the output should be
\texttt{3::nil}.  Can you write this function polymorphically so that
\texttt{x} can have type \texttt{A} for any \texttt{Set} \texttt{A}?
Use \texttt{Check} and \texttt{Eval} to confirm that your function
works as expected.

\item Write a function \texttt{isempty} to test whether a given
polymorphic list is empty or not.  Your function should return
\texttt{true} if the list is empty, and \texttt{false} otherwise.
Remember that you must do ``\texttt{Require Import List.}'' to 
import the \texttt{list} type. 

\item The \texttt{Max} file from the standard library defines a
function \texttt{max}, and the \texttt{EqNat} file defines a function
\texttt{eq\_nat}.  What do these functions do?  In each case, which
argument is the parameter of recursion?

\item There are several other ways to write the \texttt{plus}
function, besides the one we saw above, which Coq uses.  Try to define
a \texttt{plus'} function which has the same input-output behavior
as \texttt{plus}, but is written a different way (hint: one easy way
to write a different function for plus is to change which argument
is the parameter of recursion).
\end{enumerate}

\subsection{More Exercises}

The following exercises are a little more challenging.

\begin{enumerate}
\item Define a function \texttt{exp} which takes in two \texttt{nat}s,
\texttt{n} and \texttt{m}, as inputs and computes \texttt{n} to the power
\texttt{m} (so \texttt{exp 2 3} is supposed to evaluate to 8).

\item Let us use the following type for trees of natural numbers (if
you want a little more of a challenge, use a polymorphic tree type for
this exercise instead):

\begin{verbatim}
Inductive tree : Set :=
  leaf : nat -> tree
| node : nat -> tree -> tree -> tree.
\end{verbatim}

\noindent Write a recursive function that takes in a tree as input,
and returns as output a list of all the \texttt{nat}s in the tree.  It
is ok to list a \texttt{nat} more than once if it occurs in the tree
more than once.  If you know what prefix, infix, and postfix
traversals of trees are, then which one does your function implement
(and how would you implement the others)?

\item The following command defines the type \texttt{binnat} to be a
list of booleans.  This type is intended to be the type of binary
natural numbers.  Recall that a binary natural number is just a list
of bits, similarly to the way a decimal number is a list of digits.

\begin{verbatim}
Definition binnat := list bool.
\end{verbatim}

\noindent Write a recursive function that takes in a \texttt{binnat}
as input, and returns the \texttt{nat} it represents as output.  For
example, \texttt{true :: false :: false :: nil} represents 4, since
only the fours position of the number has a \texttt{true} value.
Similarly, \texttt{true :: true :: false :: nil} represents 6, since
both the fours and the twos position have \texttt{true}.

\item An unusual feature of Coq is the ability to write functions
which compute types.  Such functions operate on types similarly to
inductive data, except that pattern matching on types is not allowed.
Try to write a function \texttt{funtp} which takes in a \texttt{Set}
(i.e., computational type) \texttt{A} and a \texttt{nat} \texttt{n},
and returns the type

\begin{verbatim}
A -> ... -> A
\end{verbatim}

\noindent where there are \texttt{n} \texttt{A}s.  For example,
\texttt{funtp 4 nat} should evaluate to

\begin{verbatim}
nat -> nat -> nat -> nat
\end{verbatim}

\end{enumerate}

\section{Recursive Equations}
\label{sec:receqns}

Another, more mathematical way to write a recursive definition for a
function is using recursive equations.  Coq does not support this
notation, although some functional programming languages (e.g.,
Haskell and Caml) do.  We can define addition using recursive
equations as follows, where we take the first argument to be the
parameter of recursion:
\begin{eqnarray*}
0 + m &=& m \\
(S\ p) + m &=& S\ (p + m)
\end{eqnarray*}

\noindent This definition has one equation for each constructor with
which the parameter of recursion can be built.  Recursive calls may
only be made where the parameter of recursion has structurally
decreased, just as in Coq.

With these equation, we can prove simple facts like $2+2=4$.  Since
the numeral $2$ is really just notation (here and in Coq) for $(S\ (S\ 0))$,
what we want to prove can be written like this:
\[
(S\ (S\ 0)) + (S\ (S\ 0)) = 4
\]
Using the second equation of the recursive definition of addition, we have
\[
(\underline{S}\ (S\ 0)) + (S\ (S\ 0)) = 
\underline{S}\ ((S\ 0) + (S\ (S\ 0)))
\]
where the occurrences of $S$ matching those in the defining equation
have been underlined.  We then use the second defining equation again
to get
\[
S\ ((\underline{S}\ 0) + (S\ (S\ 0))) = 
S\ (\underline{S}\ (0 + (S\ (S\ 0))))
\]
Finally, we can use the first defining equation:
\[
S\ (S\ (\underline{0} + (S\ (S\ 0)))) = S\ (S\ (S\ (S\ 0)))
\]

Since $4$ is just notation for $S\ (S\ (S\ (S\ 0)))$, we have shown
that $2+2=4$.  This is not sufficient, of course, to show that we have
correctly defined addition.  For that, we will need to prove some
properties of addition, to see that it works as expected.  This will
be done later in these notes.

\subsection{Recursive Equations for Multiplication}

The definition of multiplication is the following, where again the
first argument is the parameter of recursion.
\[
\begin{array}{llll}
1. & 0 * m &=& 0 \\
2. & (S\ p) * m &=& m + (p * m)
\end{array}
\]

\noindent The following example showing that $2*2 = 4$ shows how the
definition works.  We use mixed notation with numerals and successor
to try to highlight how the defining equations are being used.  First
we have
\[
(S\ (S\ 0)) * 2 =
2 + ((S\ 0) * 2) 
\]
Then we get
\[
2 + ((S\ 0) * 2) = 2 + (2 + (0 * 2))
\]
Finally we have
\[
2 + (2 + (0 * 2)) = 2 + (2 + 0)
\]
Using the definition of addition, we can show that the latter term is
equal to $4$.  This chain of equalities can be written together, as
follows, where we also label the equality symbol with the number of
the defining equation that is used:

\[ \begin{array}{lll}
(S\ (S\ 0)) * 2 & =_1 & 2 + ((S\ 0) * 2) \\
\ & =_1 & 2 + (2 + (0 * 2)) \\
\ & =_2 & 2 + (2 + 0)
\end{array}
\]

\subsection{Recursive Equations for List Append}

The definition of the list append function via recursive equations is
the following (where the input lists are of type $A$, for any type
$A$):
\begin{eqnarray*}
\textit{app}\ \textit{nil}\ m &=& m \\
\textit{app}\ (n :: l_1)\ m &=& n :: (\textit{app}\ l_1\ m)
\end{eqnarray*}

\subsection{Rewriting with Equations}

Recursive definitions of the form described above can be used
computationally to rewrite terms in a finite number of steps to a
canonical form.  The form of the term is called canonical because with
recursive definitions of the kind considered here, it can be shown
that two terms are equal iff they have the same canonical form.  A
term is rewritten by finding a subterm that matches the left hand side
of one of the defining equations, and replacing that subterm with the
corresponding right hand side.  This is exactly the process we used
above to transform $2+2$ into $4$ and $2*2$ into $4$.  The study of
rewriting terms with equations is a surprisingly rich and active
research area of theoretical Computer Science, and we will touch on
further ideas from it later in these notes.

\chapter{Program Verification in Coq}

In this chapter, we are going to study techniques for proving
properties of programs written in Coq's pure functional programming
language.  The central proof methods are proof by induction and proof
by partial evaluation.  We will see how to use these methods both for
machine-checked proofs in Coq, as well as informal but rigorous proofs
in natural language (English).  Before we consider how to write
proofs, however, it is valuable to consider examples of the kinds of
properties we will prove about our programs.

\section{Example Program Properties}

The following formulas express some familiar properties of the program
\texttt{plus}, for adding unary natural numbers.  (We saw this program
defined in Coq notation in Section~\ref{sec:rec} and by recursive equations
in Section~\ref{sec:receqns}.)
\[ 
\begin{array}{lll}
x+y & = & y + x \\
(x+y)+z & = & x+(y+z)
\end{array}
\]

\noindent The first formula expresses the property called
\emph{commutativity} of addition.  The second expresses
\emph{associativity} of addition.  We have written the formulas
using variables $x$,$y$, and $z$, with the intention of expressing
that the formulas hold for all natural numbers $x$, $y$, and $z$.
This can be made more explicit by writing:
\[ 
\begin{array}{llll}
\forall\, x\,y\,:\,\texttt{nat},& x+y & = & y + x \\
\forall\, x\,y\,z\,:\,\texttt{nat}, & (x+y)+z & = & x+(y+z)
\end{array}
\]
\noindent The symbol $\forall$ is the \emph{universal quantifier}:
it is read ``for all''.  It is called universal, because we are
saying something about \emph{all} the natural numbers.  In Coq
notation, these formulas are written like this:

\begin{verbatim}
forall x y : nat, x + y = y + x 
forall x y z : nat, (x + y) + z = x + (y + z)
\end{verbatim}

\noindent In natural-language proofs, and often even in Coq, it is
allowable to omit the type declaration (``\texttt{: nat}'') from the
formula, if it is clear from the context what the types are.

Formulas, like every other object in Coq, have a type, which you can
check using the \texttt{Check} command:

\begin{verbatim}
Check forall x y : nat, x + y = y + x.
\end{verbatim}

\noindent If you evaluate this command, you will find that in Coq,
formulas have the type \texttt{Prop}.  

\subsection{More Example Properties}

Other basic arithmetic facts are also examples of program properties.
Commutativity and associativity of multiplication are examples.  An
example relating two programs is distributivity of multiplication over
addition:
\begin{eqnarray*}
x * (y + z) & = & (x * y) + (x * z)
\end{eqnarray*}
\noindent Properties like these may seem trivial, since they are so
familiar.  Nevertheless, as we will see, they are actually not
particularly trivial to prove.  And if we think about them not as
familiar mathematical facts but rather as statements about programs,
they become rather richer and more interesting.  For example,
commutativity says you can call the function in question with the
order of its arguments reversed, and the result will be the same.
This is an unusual property, since in general, programs certainly
behave differently if we scramble the order of their arguments.

Here are two properties about some of the less familiar functions we
have seen, which may thus seem less trivial.  Recall that \texttt{++} is
infix notation for appending two lists.  Also, as explored in an
exercise above, \texttt{length} is a function taking in a list and
returning its length.
\begin{eqnarray*}
(x\ \texttt{++}\ y)\ \texttt{++}\ z & = & x\ \texttt{++}\ (y\ \texttt{++}\ z)\\
\texttt{length}\ (x\ \texttt{++}\ y) & = & 
  (\texttt{length}\ x) + (\texttt{length}\ y)
\end{eqnarray*}

\noindent The first formula states that list append is associative.
If you compare it with the formula above expressing associativity of
addition, you can see the forms of these equations are very similar.
The second formula states that the length of the list you get by
appending two lists ($x$ and $y$) is equal to the sum of their
lengths.  This formula bears a certain resemblance to the formula
above for distributivity of multiplication over addition.  To express
these formulas in Coq, we have to take account of the fact that our
lists are polymorphic.  What we want to say is that for all types $A$,
and for all lists of $A$s, the property holds.  For the first
property, this can be written in Coq like this (where indentation and
spacing are purely for readability):

\begin{verbatim}
forall A : Set, forall x y z : list A, 
  (x ++ y) ++ z = x ++ (y ++ z).
\end{verbatim}

\noindent The second formula can be written similarly:

\begin{verbatim}
forall A : Set, forall x y : list A, 
  length (x ++ y) = (length x) + (length y)
\end{verbatim}

\section{Proof by Evaluation}

We will begin our study of proofs with a particularly simple and easy
kind of proof, namely proof by evaluation.  When reasoning about
programs, it sometimes happens that we can prove our \emph{goal
formula} (the formula we are currently trying to prove) simply by
running some of the programs in question.  For example, suppose we
want to prove this obvious fact:

\[ 3+4 = 2+5 
\]

\noindent We can prove this just by evaluating the program expressions
on the left and right hand sides of the equation.  By doing these
evaluations, we can transform the formula to

\[ 7 = 7
\]
\noindent This obvious fact holds, because equality is \emph{reflexive}.
That is, it satisfies the property

\[ \forall x, x = x
\]

\noindent In informal natural-language (English) proofs, proofs by
evaluation like the above are often expressed by saying something like
``by the definition of plus, both sides of the equation evaluate to
7.''  The use of reflexivity is left implicit.  If less detail is
required in the proof, one might just say, ``by arithmetic
calculation, both sides of the equation evaluate to 7.''  If more
detail is required, we can reason using recursive equations as in
Section~\ref{sec:receqns} above, to transform both sides to the same
value.

\section{Proof by Evaluation in Coq}

Let us see how proof by evaluation works in Coq.  The most common way
of writing proofs in Coq requires us to enter \emph{proof mode}, in
which special commands called \emph{tactics} can be used to try to
prove the goal formula.  To enter proof mode to prove the formula $3+4
= 2+5$, evaluate the following Vernacular command:

\begin{verbatim}
Theorem example1: 3+4 = 2+5.
\end{verbatim}

\noindent All theorems must be named in Coq.  Here, the name we have
chosen is ``example1''.  You will see the following displayed in the
top right panel of the Coq IDE (which has not displayed any output
until now):

\begin{verbatim}
1 subgoal
______________________________________(1/1)
3 + 4 = 2 + 5
\end{verbatim}

\noindent The top right panel in Coq displays the current state of a
proof when in proof mode.  Here, the message printed by Coq shows that
we are currently trying to prove 1 goal.  Coq calls this a subgoal,
because in general, it might have arisen during the proof of our
original goal formula.  The text ``(1/1)'' indicates that the formula
listed is the first of our (1) subgoals.  Then we see the actually
goal formula we are trying to prove.  To prove this formula by evaluation,
evaluate the following tactic:

\begin{verbatim}
compute.
\end{verbatim}

\noindent This tells Coq to evaluate all the program expressions
in the goal.  This is the same kind of evaluation we used with the
``Eval compute in'' Vernacular command above.  The proof state 
becomes:

\begin{verbatim}
1 subgoal
______________________________________(1/1)
7 = 7
\end{verbatim}


\noindent We can now invoke reflexivity of equality using the following
tactic:

\begin{verbatim}
reflexivity.
\end{verbatim}

\noindent Since this is enough to prove the goal formula, and since
we only had one goal to prove, Coq displays the following wonderful
and welcome message in its proof output panel:

\begin{verbatim}
Proof completed.
\end{verbatim}

\noindent To conclude the proof, we must at this point enter the
following last command:

\begin{verbatim}
Qed.
\end{verbatim}

\noindent ``Qed'' is an abbreviation for the Latin phrase ``quod est
demonstrandum'', meaning ``the thing which had to be proved''.  This
command causes Coq to exit proof mode and return to its regular mode
of interaction.  At this point, if you check the type of \texttt{example1},
you will see that its type is $3+4 = 2+5$.  This is how Coq's type theory
records the fact that it is a proof of that formula.

\section{Proof by Partial Evaluation}

Sometimes a goal formula can be proved by evaluation even when it involves
some variables.  A simple example of this is the following 

\[ \forall\, x\,:\,\texttt{nat},\ 0 * x = 0 \]

\noindent One simple way to prove a universal formula like this is to
prove that $0 * x = 0$ holds, where we temporarily assume $x$ is some
object we know nothing about, except that it is a natural number.  If
we can prove the property of this \emph{arbitrary} $x$, then we are
entitled to conclude the property holds for all natural numbers $x$.
This is because we made no assumptions about $x$ other than that it is
a natural number.  So we could plug in any natural number we want to
our proof, and the proof would still work.  We can start a
natural-language proof using this idea by saying ``Assume an arbitrary
$x$''.  Now we must show $0 * x = 0$.  This cannot be proved by
evaluation, since not all the inputs to the multiplication function
are known.  But it can be proved by \emph{partial} evaluation.  In
Computer Science, partial evaluation refers to evaluation of a program
expression containing some variables and some known values, just by
operating with the values.  In Coq, proof by partial evaluation is
typically only possible when the argument that is used as the
parameter of recursion is a known value or a partially known value.  A
partially known value is a term that starts with a constructor and
contains some variables.  Here, since multiplication's first argument
is its parameter of recursion (as can be confirmed using
\texttt{Print}, which will display the code in Figure~\ref{fig:mult}),
and since that argument is a value ($0$), we can do some partial
evaluation.  No matter what the second argument to \texttt{mult} is,
if the first argument is $0$, \texttt{mult} will return $0$.

\begin{figure}
\begin{verbatim}
mult = 
fix mult (n m : nat) {struct n} : nat :=
  match n with
  | O => 0
  | S p => m + mult p m
  end
\end{verbatim}
\caption{\label{fig:mult}Coq Definition of Multiplication}
\end{figure}

The above reasoning can be carried out in Coq as follows:

\begin{verbatim}
Theorem example2 : forall x : nat, 0 * x = 0.
intro x.
compute.
reflexivity.
Qed.
\end{verbatim}

\noindent The tactic ``\texttt{intro x}'' is how we instruct Coq to
introduce an arbitrary \texttt{x} for the universally quantified
variable.  The name we choose for this arbitrary value can be
different from the variable in the formula.  So we could just as
well have written ``\texttt{intro y}''.  The proof state we get right
after doing the \texttt{intro} is 

\begin{verbatim}
x : nat
______________________________________(1/1)
0 * x = 0
\end{verbatim}

\noindent The assumption of an arbitrary \texttt{x} of type
\texttt{nat} is listed in the \emph{context}, which is printed above
the long line.  Whenever we are reasoning under assumptions, they
appear in the context.  The remaining commands in the proof work just
as for the evaluation example above.

It should be noted that the \texttt{compute} tactic has a tendency to
unfold too many definitions.  It is generally better for partial
evaluation to use the \texttt{simpl} tactic, which is like
\texttt{compute}, but tries to leave definitions folded up when
unfolding them does not allow more computation to proceed.

\section{A First Look at Proof by Induction}

Suppose we wish to prove the formula $\forall\, x\, :\, \texttt{nat},\
x + 0 = x$.  This simple formula looks like a prime candidate for
proof by partial evaluation.  After all, it looks almost identical to
the first defining equation for addition:

\[ 0 + x = x \]

\noindent The careful reader will notice, however, that these two
formulas are not the same (even inserting the omitted quantifier into
the defining equation).  The difference is the order of the arguments.
Commutativity of addition tells us this does not matter, but at this
point in our program verification effort, we have not proved that
property.  Indeed, it turns out that we must prove the formula listed
here on the way to proving commutativity of addition.

To prove this formula, we must use proof by induction.  Proof by
induction is used to prove a formula of the form $\forall\,
x\,:\,\texttt{d},\ \phi(x)$, where \texttt{d} is an inductive datatype,
and $\phi(x)$, the \emph{body} of the universal formula, is a formula
mentioning $x$ (the lowercase Greek letter $\phi$ is pronounced
``fie'' as in ``fee fie fo fum'').  For our example, \texttt{d} is
\texttt{nat} and $\phi(x)$ is $x + 0 = x$.  Induction is a general
proof principle that works for any inductive datatype.  Here, we will
consider just the special case of induction for the unary natural number
datatype.  

To prove $\forall\, x\,:\,\texttt{nat},\ \phi(x)$ by (natural number)
induction, we do the following.  First we must prove that our property
$\phi$ holds for $0$.  That is, we must prove $\phi(0)$.  This is
called the base case of the inductive proof.  In our example,
$\phi(0)$ is $0 + 0 = 0$.  This is easily proved by evaluation.  After
we prove the base case of our inductive proof, we must prove the step
case (also called the induction or inductive case).  In the step case,
we must prove that our property $\phi$ holds for $S\ x$, under the
assumption that it holds for $x$.  This is done for an arbitrary $x$.
So we must prove $\phi(S\ x)$, under the assumption of $\phi(x)$, for
an arbitrary $x$.  The assumption of $\phi(x)$ is usually called the
induction (or inductive) hypothesis, IH.  In our example, $\phi(S\ x)$
is $(S\ x) + 0 = S\ x$, while the induction hypothesis $\phi(x)$ is
$x + 0 = x$.  The induction hypothesis lets us assume what we are
trying to prove, but only for a structurally smaller value: we assume
$\phi$ holds of $x$ when trying to prove it holds of $S\ x$.  This is
very similar to what happens in a recursive definition when we make a
recursive call.  We use the function we are trying to define, but only
for a structurally smaller value (of its parameter of recursion).

Let us see how to prove our goal formula for our current example. 
We are assuming $x + 0 = x$ (the induction hypothesis), and our
goal formula is
\[(S\ x) + 0 = S\ x\]

\noindent Using the second defining equation for addition, we can see
that the left hand of this equation can be partially evaluated:
\[S\ (x + 0) = S\ x\]

\noindent Now we can use our induction hypothesis to transform
$x + 0$ on the left hand side into $x$:

\[S\ x = S\ x\]

\noindent This is now proved by reflexivity of equality.

\section{A First Look at Proof by Induction in Coq}

Let us repeat the natural-language induction proof from the previous
section in Coq.  First, we start our theorem in the usual way:

\begin{verbatim}
Theorem example3: forall x : nat, x + 0 = x.
\end{verbatim}

\noindent Coq requires us to specify that we are doing induction on
variable $x$, using this tactic:

\begin{verbatim}
induction x.
\end{verbatim}

\noindent At this point the proof state looks like this:

\begin{verbatim}
2 subgoals
______________________________________(1/2)
0 + 0 = 0


______________________________________(2/2)
S x + 0 = S x
\end{verbatim}

\noindent There are two subgoals: the first is for the base case, and
the second is for the step case.  They look just the way we expect
(although we see in the second subgoal that Coq's parsing conventions
have allowed parentheses to be omitted where we wrote them above).
Note that for better readability, Coq does not list the assumptions in
the second subgoal.  We will see them when we start working on that
subgoal.  The first subgoal listed is always the one we are currently
working on.  To prove $0+0 = 0$, we use proof by evaluation with
the \texttt{compute} tactic.  This transforms the proof state to be
the following

\begin{verbatim}
2 subgoals
______________________________________(1/2)
0 = 0


______________________________________(2/2)
S x + 0 = S x
\end{verbatim}

\noindent At this point, invoking reflexivity finishes the proof of the
first subgoal, and our proof state becomes:

\begin{verbatim}
1 subgoal
x : nat
IHx : x + 0 = x
______________________________________(1/1)
S x + 0 = S x
\end{verbatim}

\noindent Here we are presented with the step case of our inductive
proof.  We have an induction hypothesis called \texttt{IHx}, stating
that $x + 0 = x$.  Our goal formula is as expected.  The assumption
and goal formula are expressed with respect to an arbitrary
\texttt{nat} $x$, which is declared in the context (the part above the
line).  At this point, we wish to do proof by partial evaluation.  As
remarked above, it is usually best to use \texttt{simpl} for this
rather than \texttt{compute}.  If we use the \texttt{simpl} tactic, we
get

\begin{verbatim}
1 subgoal
x : nat
IHx : x + 0 = x
______________________________________(1/1)
S (x + 0) = S x
\end{verbatim}

\noindent We now must use our induction hypothesis \texttt{IHx}.  The tactic
to do this is called \texttt{rewrite}:

\begin{verbatim}
rewrite IHx.
\end{verbatim}

\noindent This tactic transforms the left hand side of the formula
associated with \texttt{IHx} into the right hand side, anywhere it
finds the former in the goal formula.  So it transforms \texttt{x+0}
into just \texttt{x}, in the single place where \texttt{x+0} occurs
(on the left hand side of the goal formula).  We get

\begin{verbatim}
1 subgoal
x : nat
IHx : x + 0 = x
______________________________________(1/1)
S x = S x
\end{verbatim}

\noindent This is now concluded using \texttt{reflexivity} (and then
\texttt{Qed}).

\section{Exercises}

Do the following in Coq, using (only) the tactics discussed in
this chapter so far:

\begin{enumerate}
\item Prove the following obvious facts by evaluation (remember to do
``\texttt{Require Import List}'' to load the list module for the second
formula):
\begin{verbatim}
10 * 10 = 5 * 20.
(1::2::nil) ++ (3::nil) = (1::nil) ++ (2::3::nil).
\end{verbatim}

\item Prove the following simple facts by partial evaluation.  Note
that to assume several arbitrary objects at once, such as \texttt{A}
and \texttt{x} in the first formula, you can use the tactic
``\texttt{intros A x.}'', instead of more verbosely using
``\texttt{intro A.}'' and then ``\texttt{intro x.}''.
\begin{verbatim}
forall A : Set, forall x : list A, nil ++ x = x.
forall A : Set, forall x y z : A, 
  (x::y::nil) ++ (z::nil) = (x::nil) ++ (y::z::nil).
\end{verbatim}

\item Prove this fact by induction: $\forall\, x\, :\, \texttt{nat},\ x * 0 = 0$.

\item Prove this fact by induction: $\forall\, x\, y\, :\,
\texttt{nat},\ x + (S\ y) = S\ (x + y)$.  This is another fact used in
the proof of commutativity of addition.  Note that in the step case,
you will need to rewrite using your induction hypothesis, which will
be universally quantified (if you do your proof in the obvious way).
Rewriting works just fine with universally quantified equations, so
you can use ``\texttt{rewrite IHx}'', as in the examples above, to do
the rewriting you need.

\end{enumerate}

\section{More Exercises}

Do the following in Coq, again using (only) the tactics discussed in
this chapter so far:

\begin{enumerate}
\item Prove the formula
\begin{verbatim}
forall x : nat, 0 - x = 0
\end{verbatim}
\item Prove the formula
\begin{verbatim}
forall x : nat, x - 0 = x
\end{verbatim}
\item Prove associativity of addition.
\item Prove associativity of list append.
\end{enumerate}

\section{Case Study: Binary Search Trees}
                                                                             
The following exercises concerning binary search trees are based on
the file bst.v.

\begin{enumerate}

\item Prove the admitted theorem intree\_insert.  This states that if y
is in tree t or y equals x, then y is in the tree resulting from
inserting x into t; and vice versa.  The ``vice versa'' is implemented
by using \texttt{<-->}, which is for ``if and only if''.

\item This exercise builds up to the theorem that if a tree t is a
binary search tree, then searching for a piece of data x in t using
the intree function (which searches the whole tree) is equivalent to
searching for x in t using the inbst function (which can search more
efficiently, using the binary search tree approach).
                                                                              
The file bst.v contains a statement of this theorem, called
bst\_inbst.  The proof I have for this theorem is quite involved,
using 6 helper lemmas.  I have included proofs for two of these helper
lemmas.  The other lemmas have been admitted.
                                                                              
Before you try the following, you should make sure you understand
intuitively why the theorem is correct.  It will not be easy to prove
this in Coq if you do not really understand why it is true.  Draw some
pictures of binary search trees with upper and lower bounds to test
your understanding.  Make sure you see how the helper lemmas I have
written will be used in the proof.

\begin{enumerate}                                                
\item Prove the helper lemma leq\_total.

\item Prove the theorem bst\_inbst.  You do not need to prove the
admitted helper lemmas.  You may use any of them that end up being
needed in the proof of bst\_inbst.
                                                                               
\item Prove some of the other admitted helper lemmas.

\end{enumerate}
\end{enumerate}

\end{document}

